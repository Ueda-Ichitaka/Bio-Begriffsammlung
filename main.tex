\documentclass{article}
\usepackage[utf8]{inputenc}
\usepackage[english]{babel}
\usepackage{geometry}
\usepackage{minted}
\usepackage{tabularx}
\usepackage{listings}
\usepackage{bussproofs}
\usepackage{multicol}
\usepackage{upgreek}
\usepackage{graphicx}
\usepackage{color}
\usepackage{amsmath}

%\lstset{language=JVMIS}
\usemintedstyle{borland}
%\geometry{left=10mm, top=20mm, bottom=10mm}
\geometry{left=10mm, top=10mm, bottom=10mm}
\definecolor{mygreen}{rgb}{0,0.53,0}

\expandafter\def\csname PYGborland@tok@na\endcsname{\def\PYGborland@tc##1{\textcolor[rgb]{1.00,0.00,0.00}{##1}}}

\pagenumbering{gobble}

\lstset{
  keywordstyle=\bfseries,
  morekeywords={
    aload,aload_0,areturn,arraylength,astore,astore_0,getfield,getstatic,goto,
    goto_w,iadd,iand,iconst_0,idiv,if_icmpeq,if_icmpge,if_icmpgt,if_icmple,
    if_icmplt,if_icmpne,ifeq,ifge,ifgt,ifle,iflt,ifne,ifnonnull,ifnull,iinc,
    iload, iload_0,imul,ineg,instanceof,invokedynamic,invokeinterface,
    invokespecial,invokestatic,invokevirtual,ior,ireturn,ishl,ishr,istore,
    istore_0,isub,iushr,ixor,ldc,new,nop,pop,pop2,putfield,putstatic,ret,return,
    swap
  },
  sensitive=false,
  morecomment=[l]{##},
  commentstyle=\bfseries\color{mygreen},
  %morecomment=[s]{/*}{*/},
  %morestring=[b]",
}

\begin{document}

\vspace{-1mm}

\section{A}
\begin{minted}[linenos,numberblanklines=false,tabsize=4,breaklines,breakindent=25pt]{haskell}
Beispielbegriff :: Kurzerklärung -- Lange erklärung 

Aberration :: Chromosomen-Aberration -- Aneuploidie (z. B. Monosomie, Trisomie), Polyploidie (z. B. Triploidie) & Inversion, Deletion, Translokation, Duplikationen

Abiogenese -- Urzeugung, Erklärung der Entstehung von Leben aus Unbelebtem. (erste Nucleotide vor 4.2 Mrd. Jahren)

Actin -- Teil des Cytoskeletts, bildet spiralige Fäden, an denen Myosinmotoren entlangwandern, Baustein der Muskeln, kein Heterodimer

Actomyosin -- Überbegriff für das actinbasierte Bewegungssystem bestehend aus einer Actin"straße" und einem Myosin"motor"

Adhäsom -- alle Proteine die zur Verankerung der Zelle benötigt werden

adaptive Radiation -- schnelle Artbildung durch Besiedlung verschiedener ökologischer Nischen in einem neuen Lebensraum

Aggregation -- Entstehung von Vielzelligkeit durch "Haufenbildung" einzelliger Formen

aktiver Transport -- Transport eines Moleküls über eine Membran entgegen dem chemischen Gradienten unter Aufwand von Energie

Aktualitätsprinzip -- Rekonstruktion der geologischen Vorzeit aufgrund der Gesetze der Jetztzeit (geht auf Lyell zurück)

Aktinfilamente -- haben einen Durchmesser von 7 nm. Das Filament besteht aus einer ver- drillten Kette von globulären Aktinmole- külen. Aktinfilamente sind flexibler und kürzer als Mikrotubuli. Die Ge- samtlänge aller Actin-filamente in einer Zelle ist etwa 30mal größer als die aller Mikrotubuli.

Allel -- Variante eines Gens, z.B. bei einer Mutation entsteht aus dem Wildtyp- ein Mutanten-Allel

allopatrisch -- Variante der Artbildung, bei der die Subpopulation zuerst geographisch getrennt werden und danach genetisch allmählich so unterschiedlich werden, dass irgendwann eine Fortpflanzungsbarriere entsteht.

Allopolyploidie -- Hybridisierung mit anschliessender Verdopplung des Genoms, so dass eine fertile Nachkommenschaft entsteht. Häufiger Mechanismus der pflanzlichen Artbildung

Altruismus -- Gegenstück zu Egoismus, ein Individuum handelt zum Wohle der Anderen

alternatives Splißing -- Beim alternativen Splicing entscheidet sich erst während des Spleißvorgangs, welche RNA-Sequenzen Introns und welche Exons sind. Aus ein und derselben DNA-Sequenz und dementsprechend ein und derselben prä-mRNA können mehrere verschiedene reife mRNA-Moleküle und durch deren Translation auch mehrere unterschiedliche Polypeptide oder Proteine gebildet werden.

amphipathisch -- Moleküle, die hydrophobe und hydrophile Bereiche vereinigen (z.B. Membranlipide)

Antikörper :: bilden Aggregate mit fremden Molekülen, Viren, Bakterien -- Antikörper sind Proteine, die sehr fest an ihre Ziel-strukturen (Antigene) binden. Sie werden von Vertebrate als Antwort auf eine Infektion produziert. Jedes Antikörpermolekül besteht aus zwei identischen leichten Ketten und zwei identischen schweren Ketten, so da die beiden Antigenbindungsstellen identisch sind. Antiköroper werden von einer Gruppe weißer Blutkörperchen, den B-Lymphozyten (B-Zellen) hergestellt. Herstullung in Tieren: z.B. einer Maus wird wiederholt ein Anttigen A gespritzt (im Abstand von mehreren Wochen), das stimuliert B-Zellen, viele leicht verschiedene Antikörper (polyklonale Antikörper) werden hergestellt und können dann aus dem Blutserum gefiltert werden.

Analogie -- eine durch ähnliche Funktion entstandene Ähnlichkeit der Gestalt (Synonym: Konvergenz), Gegensatz zur Homologie, wo die Ähnlichkeit auf Verwandtschaft (Abstammung) zurückgeht.

Aneuploidie -- einzelne Chromosomen sind in einer vom üblichen Chromosomensatz abweichenden Anzahl vorhanden, also entweder öfter, weniger oft oder ganz fehlen

Angiosperm :: Bedecktsamer. Blütenpflanzen -- die größte Klasse der Samenpflanzen. Sie unterscheiden sich von den Nacktsamern darin, dass in ihren Blüten die Samenanlagen von einem Fruchtblatt bzw. Fruchtknoten umschlossen und darin geschützt („bedeckt“) liegen. Käferblütigkeit -> Nektarblütigkeit -> Exklusive Bestäubung (z.B. spezielle Wespen)

Anticodon -- komplementäre Sequenz aus drei Nucleotiden auf der tRNS, das sich mit dem Codon auf der mRNS paart.

Aquaporine -- Carrier für Wasser, erhöhen die Wasserpermeabilität von Membranen für schnelle Transportreaktionen

Archaeen -- drittes Reich der Lebewesen neben Pro- und Eukaryoten. Die Archaeen kommen in Extremstandorten vor und entsprechen wohl den ältesten Lebensformen.

Artkonstanz -- die Vorstellung, dass Arten unveränderlich sind, häufig verbunden mit der Vorstellung, dass sie in dieser Form von einem göttlichen Wesen geschaffen worden seien. Es gibt christliche, aber auch muslimische Formen dieser kreationistischen Vorstellung.

Atmungskette -- Komplex aus Membranproteinen in der inneren Mitochondrienmembran, wo bei der Atmung ein Protonengradient entsteht, der dann die ATP-Synthase treibt.

ATP-Synthase -- eines der wichtigsten Proteine dieser Erde, das an der inneren Mitochondrienmembran oder der Thylakoidmembran sitzt und einen elektrischen Gradienten (aus Protonen) dazu benutzt, das energiereiche ATP herzustellen, mit dem zahlreiche energieverbrauchende Reaktionen des Lebens angetrieben werden.
    
Apoptose -- programmierter Zelltod durch anschwellen, kollabierendes Cytoskelett    

apikal -- gegenseitig zu basal, frei und der luft oder einer flüssigkeit ausgesetzt

Adhärenzkontakt -- koppeln Zellen dauerhaft aneinander. Sie werden durch Cadherine gebildet, die homophile Bindungen mit Cadherinen in der Nachbarzelle eingehen. Intrazellulär ist jedes Cadherin über Proteinlinker an Aktinfilamente angebunden. Es entsteht einAdhäsionsgürtel aus gebündelten Aktinfilamenten, der das Epithel stabilisiert.

Adenylatcyclase :: stellt cAMP her -- Zielprotein von G-Protein Second Messengern

Auflösung (Mikroskop) -- kleinster Abstand zwischen zwei Punkten, sodass man sie unterscheiden kann. Auflösung:  d = lambda / A, A = n * sin(alpha), n = Brechnungsindex, lambda = Wellenlänge
    
\end{minted}
\newpage


\section{B}
\begin{minted}[linenos,numberblanklines=false,tabsize=4,breaklines,breakindent=25pt]{haskell}
Benthos -- festsitzende Flora der Gewässer, wegen der Abhängigkeit vom Licht auf die Küstenlinie beschränkt (Gegenbegriff: Plankton)

Binomiale Nomenklatur -- Bezeichnung aller Lebensformen über einen Gattungs- und einen Artnamen. Von Carl v. Linné allgemein eingeführt.

Biofilmtheorie -- Theorie, dass Leben an katalytische wirksamen Oberflächen von Pyriten entstand (alternativ zur Ursuppentheorie von Oparin und Haldane)

Biogenetisches Grundgesetz -- Modell von Häckel, wonach ein Lebewesen in der Individualentwicklung die Stammesentwicklung wiederholt

Barr Körperchen :: --

Basallamina -- Trennt Zellen vom umgebenden Gewebe, Wirkt als Adhäsionsmatrix für alle Epithelien, Umhüllt jede Muskelzelle. Unterliegt jedem Epithel und trennt dieses vom unterliegenden Bindegewebe (Blasenbildung nach Verbrennung!). Ist maßgeblich für die Filtrationsfunktion der Niere.

B-Lymphozyten (B-Zellen) -- stellen Antikörper her. Jede B-Zelle trägt auf ihrer Oberfläche einen Rezeptor für die Erkennung des Antigens. Bindet das Antigen an den Rezeptor werden die B-Zellen stimuliert. Dies führt zur Zellteilung und Produktion gorßer Mengen löslicher Antikörpern

Biotrophie -- der Parasit bildet Haustorien und steuert die Wirtszellemit chemischen Signalen (Effektoren) so um, dass sie den Parasiten mit Nährstoffen versorgt, oder dabei zu sterben. Dazu wird die basale Immunität des Wirts (PTI) ausgeschaltet. Fortgeschrittene Evolution
\end{minted}
\newpage


\section{C}
\begin{minted}[linenos,numberblanklines=false,tabsize=4,breaklines,breakindent=25pt]{haskell}
Carrier -- Membranproteine, die den Durchfluss von Molekülen ermöglichen, die sonst nicht hindurchgelangen würden

Caspasen -- Familie proteolytischer Enzyme, die an der Apoptose beiteiligt sind

Cadherin -- vermitteln eine Ca 2+-abhängige, homophile Bindung zwischen einzelnen Zellen. Unterschiedliche Cadherine werden oft gewebespezifisch exprimiert (z.B. N-Cadherin im Nervensystem). \textbf{Cadherine bilden Adhärenzkontakte}

Centrosom -- In vivo haben die meisten Mikrotubuli ihren Ursprung in einem Punkt in der Nähe des Zellkerns. Diese Struktur wird als Mikrotubuli- organisationszentrum (MTOC) oder Centrosom bezeichnet. Hauptbestandteile: Centriolen, pericentrioläres Material (PCM)

Centriole -- zylinderförmige Struktur aus Mikrotubuli-Triplets, windradähnliche Anordnung von Triplettmikrotubuli im Centrosom von tierischen Zellen

Centromer -- chromosomaler Bereich, der die Chromosomenbewegung während der Mitose reguliert

Chitin -- Baustoff der Zellwand von Pilzen und Insekten, Polymer, das aus Acetyl-Glucosamin besteht

Chlorophyll -- Blattgrün, Pflanzenfarbstoff, besteht aus einem hydrophilen Porphyrinring, wo die Lichtreaktion der Photosynthese stattfindet und einem hydrophoben Phytolschwanz, mit dem der Farbstoff an der Thylakoidmembran verankert ist.

Chromatin -- verpackte DNS im Zellkern, aus Nucleosomen (Histone + DNA) (Die Nucleosomen werden mit Hilfe der Nichthiston-Proteine dichter gepackt.), zum Ablesen (Transkription) muss sie ausgepackt werden, Name von chromos = Farbe, da man Chromatin histochemisch anfärben kann. (bead on a string). Heterochromatin (Chromosomen Abschnitte die sich nie entwickeln, dicht gepackt und transpriktionell inaktiv) Euchromatin (locker gepackt und transpriktionell aktiv)

Chromatid :: elementarer Teil eines Chromosoms -- Vor der DNA-Replikation besteht jedes Chromosom aus einem einzigen DNA-Molekül, dem Chromatid. Nach der Replikation besteht jedes Chromosom aus zwei identischen DNA-Molekülen, den Schwesterchromatiden

Codon -- "Wort" auf der mRNS aus drei Nucleotid"buchstaben", das eine Aminosäure "bedeutet"

Codominanz -- Allele eines Gens, welche beide unabhängig voneinander ihren Phäntotyp exprimieren.

Cilien -- haarähnliche Strukturen (z.B. Flimmerepithel) zur Bewegung, eher kurz und in großer Anzahl auf Zelle

Cristae -- Einstülpungen der inneren Mitochondrienmembran, wo die ATP-Synthase sitzt.

Cyclin -- Protein, das in einem Komplex mit einer Kinase den Zellzyklus weiterführt. Nach dem Zellzyklusschritt wird es abgebaut, um eine Dauerteilung zu. verhindern. Cyclin bildet einen Komplex mit MPF 

Chaperone :: falten Proteine -- 

C4-Pflanzen: Dunkelreaktion und Lichtreaktion in verschiedene Zellen aufgetrennt. -> Strategie zur Unterdrückung der Photorespiration. Vor allem bei viel Licht und geringem Kohlendioxidgehalt wichtig.

Chargaff-Regel -- Analyse verschiender DNA-Proben ergab: Anzahl Adenin = Anzahl Thymin, Anzahl Guanin = Anzahl Cytosin

\end{minted}
\newpage


\section{D}
\begin{minted}[linenos,numberblanklines=false,tabsize=4,breaklines,breakindent=25pt]{haskell}
degenerierter Code -- 20 Aminosäuren sind 64 mögliche Basentripletts zugeordnet, jedes Triplett entspricht eindeutig einer Aminosäure, aber eine Aminosäure kann von mehreren Tripletts kodiert sein.

Destruenten -- in Ökosystemen Organismen, die von den Überresten anderer Organismen leben.

Desmosom -- Diese Verbindung wirkt wie ein Schweißpunkt; die starken Intermediärfilamente einer Zelle werden mit denen der Nachbarzelle hier verankert.

Deszendenztheorie -- Erklärung von biologischer Ähnlichkeit aufgrund von Abstammung (widerspricht der Artkonstanz)

Deuterostomier -- Gruppe der Metazoa, wo der Urmund zum After wird, der Mund entsteht sekundär als Durchbruch gegenüber. Die "Wirbeltiere"

differentielle Genexpression -- Modell der Entwicklung, wonach alle Zellen zwar denselben Satz an Genen erhalten, aber unterschiedliche Teile davon aktivieren.

diploid -- Zelle enthält zwei Kopien jedes Chromosoms, eine vom Vater, eine von der Mutter

Dynamische Instabilität -- Charakteristikum einzelner Mikrotubuli ::  Hierbei können Phasen der Polymerisation durch abrupte Abbauphasen unterbrochen werden und der Mikrotubulus so zwischen Phasen des Wachstums und der Verkürzung hin- und herpendeln.
\end{minted}
\newpage


\section{E}
\begin{minted}[linenos,numberblanklines=false,tabsize=4,breaklines,breakindent=25pt]{haskell}
Egoistisches Gen -- Konzept von Richard Dawkins, das die Evolution von Altruismus erklären soll. Gene, die ihre eigene Vermehrung fördern, indem sie altruistisches Verhalten fördern, "handeln" im eigenen Interesse und werden von Dawkins als "egoistisch" (selfish) bezeichnet. Durch die sprachliche Psychologisierung der Genetik hat dieses Konzept viel Verwirrung gestiftet.

Ektoderm -- Außenschicht des Metazoen-Keims, dazu zählen Haut, Nervengewebe, Magen. Gegensatz Endoderm und Mesoderm

Endocytose :: Aufnahme von Substanzen aus dem Extrazellularraum in die Zelle -- aktiver Transportvorgang bei dem die zu transportierenden Stoffe in membranumschlossene Vesikel eingeschlossen werden. Dabei werden die Inhaltsstoffe ins Innere der Zelle transportiert

Endoplasmatisches Reticulum -- Membransystem, das das Innere eukaryotischer Zellen durchzieht. Stapel des ER bilden die Kernhülle, vom ER knospen Vesikel ab, die Stoffe zum Golgi-Apparat bringen, von wo sie weiter transportiert werden

Endosomen :: Sortierung von aufgenommenem Material -- Vesikel, die durch Einstülpung an der Zellmembran entstehen und Dinge in die Zelle aufnehmen.

Endosymbiontentheorie -- auf Mereschkowskij und Margulis zurückgehende Theorie, dass Mitochondrien und Plastiden durch Endosymbiose von freilebenden Prokaryoten entstanden sind, dies erklärt die eigene (prokaryotische) DNS dieser Organellen, ihre eigenen (prokaryotischen) Ribosomen und ihre doppelte Membran.

Endproduktrepression -- Fall der Gensteuerung, bei dem ein Operon durch das (gebildete) Endprodukt abgeschaltet wird. Beispiel: Tryptophan-Synthese

Epigenese -- Modell der Entwicklung, wo die Gestalt des Organismus erst durch Wechselwirkungen der Teile des Embryo entsteht.

Epistase -- Die Fähigkeit eines Gens den Phänotyp eines anderen Gens zu verändern bzw. zu überlagern.

Epithelgewebe -- Epithelschicht ist polarisiert und hat 2 Seiten. Die apikale Seite ist frei und der Luft oder einer wäßrigen Flüssigkeit ausgesetzt. Die basale Seite ruht meist auf einem Bindegewebe mit Basallamina.

Epigenetik -- Die Fähigkeit eines Gens zur Expression ändert sich abhängig davon, wie oft es zuvor exprimiert wurde (eine Art "Gentraining"). Zellulär liegen Änderungen der Kernarchitektur, Methylierung der DNS und Deacetylierung der Histone zugrunde. Kann in manchen Fällen auch an die nächste Generation weitergegeben werden.

Eukaryoten -- Organismen, bei denen die DNS in einem Zellkern vom Cytoplasma getrennt ist: Pilze, Pflanzen und Tiere

Exozytose :: Ausschleusen von Substanzen aus der Zelle in den Extrazellularraum -- aktiver Transportvorgang bei dem die zu transportierenden Stoffe in membranumschlossene Vesikel eingeschlossen werden und durch Membranfusion wieder freigegeben werden. Dabei werden die Inhaltsstoffe nach außen (in den Extrazellularraum) abgegeben

Exon shuffling :: Hypothese -- Durch Rekombination zwischen verschiedenen Intronsequenzen (Intron) werden neue Kombinationen von Exonen (Exon) geschaffen. Zwei oder mehr exons aus verschiedenen Genen rekombinieren, oder das selbe Exon dupliziert um neue Intron-Exon Strukturen zu schaffen.
\end{minted}
\newpage


\section{F}
\begin{minted}[linenos,numberblanklines=false,tabsize=4,breaklines,breakindent=25pt]{haskell}
Fluid-Mosaic Model ::  von Singer und Nicholson -- beschreibt die Membran als flüssige Matrix, worin Proteine mosaikartig eingebettet sind.

Flagellen -- besitzen ähnliche Struktur wie Cilien, aber länger, und weniger pro Zelle, zur Fortbewegung der Zelle (Spermium) 

Fokalkontakte -- dynamische Komplexe aus über 50 verschiedenen Proteinen. Hier binden Membranrezeptoren an extrazelluläre Proteine und intrazellulär über Bindeproteine an das Aktin-Cytoskelett.

F-Plasmid :: Fertilitätsplasmid --  Plasmid, das Bakterien die Fähigkeit zur Konjugation (horizontaler Gentransfer) verleiht. Das F-Plasmid ermöglicht einen gerichteten Gentransfer vom Spender (dieser besitzt den F-Faktor, wird auch als F+ bezeichnet) zum Empfänger (F-)

\end{minted}
\newpage


\section{G}
\begin{minted}[linenos,numberblanklines=false,tabsize=4,breaklines,breakindent=25pt]{haskell}
Gamet -- Keimzelle (Eizelle, Spermium)

Gen -- Genetischer Faktor (Region der DNA), der Merkmal bestimmt

Genkonversion -- kaputte DNA (mit Loch im Strang) wird aufgefüllt mit DNA einer "Vorlage". Vorlage kann sein: Schwesterchromatid, homologes Chromosom, oder aus einer Sequenz wo ganz anders die gleich ist. Genkonversion kann durch Mismatch Reparatur von Heteroduplex- DNA während der Rekombination entstehen. Genkonversion ist an die Rekombination während der Meiose gekoppelt.

G0-Phase -- zeitweise oder stabile Unterbrechung des Zellzyklus (G steht für gap), wenn eine Zelle sich differenziert (Nervenzellen sind in G0)

Generation -- Folge von >1 Mitose zwischen Meiose und Befruchtung

Generationswechsel -- regelmäßige Abfolge von Generationen, die sich in ihrer Fortpflanzung oder Kernphase unterscheiden (für Pflanzen die Regel).

genetic drift -- zufällige Einflüsse auf die genetische Zusammensetzung einer Population, die nichts mit fitness zu tun haben. Spielt vor allem bei kleinen Populationen eine Rolle.

Gelege -- Gesamtheit der von einem Tier an einer Stelle abgelegten Eie

Glucose -- Baustein von Stärke und Cellulose, kann durch Wasserabspaltung verknüpft werden. Wenn die OH-Gruppe dabei nach unten zeigt, entsteht eine a-Verknüpfung (die Zuckerreste sind gleich orientiert, es entsteht eine Spirale: Amylose), wenn sie nach oben zeigt, entsteht eine b-Verknüpfung (jeder zweite Zuckerrest ist gedreht, es entsteht eine gestreckte Konfiguration, Cellulose).

Glycocalyx -- "Zuckerkelch", die zuckerhaltige Außenschicht tierischer Zellen (entsteht durch Glycolipide und Glycoproteine). Periphere und integrale Membranproteine

Glycolipid -- Zuckermoleküle gebunden an Phospolipid-Membran (nur ausserhalb der Zelle)

Golgi-Apparat :: baut Proteine um -- dynamisches Organell aus Membranstapeln, wo Proteine, die vom ER kommen, prozessiert werden

Gradualismus -- die Auffassung, dass Evolution in kleinen Schritten gleichmäßig voranschreitet (Darwins Auffassung)

Gründereffekt -- Sonderform der genetic drift, bei der eine kleine Individuengruppe eine neue Population begründet, wodurch zufällige Abweichungen der Allelhäufigkeiten dann große Folgen haben

Gruppenselektion -- Selektion einer Gruppe von Individuen aufgrund einer Eigenschaft, die alle Individuen besitzen (z.B. Schwarmbildung)
\end{minted}
\newpage


\section{H}
\begin{minted}[linenos,numberblanklines=false,tabsize=4,breaklines,breakindent=25pt]{haskell}

Haushaltsproteine -- Proteine die in allen Zellen eines Organismus exprimiert werden (-> Haushaltsgene)

Haploid -- Zelle enthält nur ein Chromoson eines paares (Gameten)

Heterochronie -- zeitliche Verschiebung von Entwicklungsvorgängen, führt zu Veränderungen des Bauplans

Heterocyste -- stickstoff-fixierende Zelle der Cyanobacterien ("Blaualgen"), die keine Photosynthese
betreibt und von den Nachbarn mit Assimilaten versorgt werden muss.

Heterodimer -- ein Molekül oder ein Molekülverbund, der aus zwei unterschiedlichen Untereinheiten, den sog. Monomeren, besteht.

Heterozygot -- Individuum mit zwei unterschiedlichen Allelen an einem bestimmten Lokus

Histone -- Proteine, die an die DNS binden und sie zu Knäueln (Nucleosomen) "verpacken"

Homologiekriterien -- Kriterien für eine evolutionäre Verwandtschaft - Lage, spezifische Qualität und Kontinuität

Homologie -- homologe Merkmale gehen auf Merkmale des gemeinsamen Vorfahren zurück, sie sind also gleichwertig bezüglich ihrer stammesgeschichtlichen Herkunft.

Homozygot -- Alle Allele eines Individuums Individuum mit zwei gleichen Allelen an einem bestimmten Lokus

Hybridisierung -- Paarung spiegelbildlicher DNS- oder RNS-Abschnitte zu stabilen Doppelsträngen

Histamin

Heterochromatin ::  verdichtetes Chromatin im Zellkern -- besteht aus sich nie entwindenden chromosomen abschnitten. dicht gepackt und transkriptionell inaktiv

Hemidesmosom -- verankert die Zelle an der Basallamina. Hier werden die Intermediärfilamente über Integrinmoleküle in der Zellmembran mit der ECM der Basallamina verbunden.

Hfr-Stamm :: (Abk. für engl. high frequency of recombination) -- Bakterienstamm, der sich durch eine hohe Rekombinationsrate auszeichnet. Beispiel für Hfr-Stämme sind Bakterien mit F-Plasmid.

\end{minted}
\newpage


\section{I}
\begin{minted}[linenos,numberblanklines=false,tabsize=4,breaklines,breakindent=25pt]{haskell}
Intelligent Design -- pseudowissenschaftliche Spielart des Kreationismus, wo komplexe Strukturen als Spuren eines "intelligent design" betrachtet werden.

Intron -- nicht-kodierende Abschnitte der DNS, die in die kodierende Sequenz eingefügt sind und vor der Translation durch Spleissen herausgeschnitten werden. Diese Stücke steuern dann das Ablesen weiterer Gene, sind also eine Art Sprache für die gegenseitige Kommunikation der Gene.

Integrin -- Zellrezeptoren, die die Bindung an Moleküle der ECM vermitteln. Die Integrinfamilie besteht aus etwa 20 Mitgliedern. Sie bilden untereinander Heterodimere, die aus a- und ß-Intergrinen bestehen und dann selektiv bestimmte ECM-Liganden (Fibronektin, Laminin, Kollagen etc.) erkennen.

Intermediärfilamente -- im Cytoplasma einer Zelle gelegene Strukturen aus Proteinen, die der Erhöhung der mechanischen Stabilität der Zelle dienen. Neben Mikrotubuli und Mikrofilamenten Teil der strukturellen Hauptkomponenten des Zytoskeletts. Mit Durchmesser von etwa 10 Nanometern zwischen den Mikrofilamenten (7 nm) und den Mikrotubuli (20 bis 30 nm). Bei Pflanzen kommen Intermediärfilamente nicht vor.

Immunfluoreszenzmikroskopie -- Werden Floureszenzfarbstoffe an einen Antikörper gebunden, können damit bestimmte Proteine innerhalb der Zelle lokalisiert werden

\end{minted}
\newpage

\section{J}
\begin{minted}[linenos,numberblanklines=false,tabsize=4,breaklines,breakindent=25pt]{haskell}

\end{minted}
\newpage


\section{K}
\begin{minted}[linenos,numberblanklines=false,tabsize=4,breaklines,breakindent=25pt]{haskell}
K-Strategie -- ökologische Strategie von Organismen, die auf nachhaltige Nutzung der Resourcen bei langsamer Vermehrung abzielt

Käferblütigkeit -- Seerosen, Holunder, Dolden-, Korbblütler). Angiospermen.

Kambrische (Arten-)Explosion -- fast gleichzeitiges, erstmaliges Vorkommen von Vertretern fast aller heutigen Tierstämme. Fand zu Beginn des Kambriums statt

Katastrophentheorie -- Versuch von Cuvier, die Weiterentwicklung von Fossilien mit dem Dogma der Artkonstanz zu verbinden. Katastrophen seien von Neuschöpfung gefolgt

Kariotyp -- Alle Chromosomen eines Individuums

Kern-Plasma-Relation -- Schwellenwert von Plasma pro DNS, bei dessen Überschreiten eine Zellteilung eingeleitet wird.

Kernlamina -- dichter, fibrillärer Verbund, der weitgehend direkt unter der Kernhülle liegt und etwa 30 bis 100 nm stark ist. Sie enthält Intermediärfilamente und Proteine, die mit der inneren Membran der Kernhülle verbunden sind. Neben einer Stützfunktion spielt die Lamina eine Rolle bei Vorgängen wie der Regulation der DNA-Replikation und der Zellteilung sowie bei der Chromatinorganisation.

Kinesin -- Klasse von Mikrotubulimotoren, die in der Regel durch Spaltung von ATP zum Pluspol des Mikrotubulus wandern

Koevolution -- gemeinsame evolutionäre Entwicklung von nicht verwandten Organismen. Paradebeispiel wären Blütenpflanzen und Hautflügler.

Kompartiment -- durch eine Membran umschlossener Reaktionsraum im Innern einer Zelle

Kormophyten :: alle höheren Landpflanzen (keine Moose). Urfarne - Erfinder des Kormus -- Als Kormus bezeichnet man den Vegetationskörper einer Pflanze, die in Sproßachse, Blatt und Wurzel gegliedert ist. Kormophyten sind alle Pflanzen, die diese Gliederung aufweisen, oder nachweislich von Vorfahren mit dieser Gliederung abstammen.

Konsumenten -- in der Ökologie die Organismen, die durch Fressen anderer Organismen ihre Energie gewinnen

Kontinuitätsprinzip -- natura non saltat, Erklärung von Evolution über kleine Schritte

Konvergenz -- Ähnlichkeit von Strukturen, die nicht auf Verwandtschaft, sondern ähnlicher Funktion beruht

Kopplung -- wenn die Allele zweier Merkmale gemeinsam vererbt werden, spricht man von genetischer Kopplung. Deutung: die Gene für die beiden Merkmale liegen auf einem Chromosom.

Kormus :: Grundbauplan aller höheren Landpflanzen (Farne und Samenpflanzen) aus stabförmigen Elementen (Telomen), die aus Leitgewebe, Parenchym und Epidermis bestehen -- Als Kormus bezeichnet man den Vegetationskörper einer Pflanze, die in Sproßachse, Blatt und Wurzel gegliedert ist. Kormophyten sind alle Pflanzen, die diese Gliederung aufweisen, oder nachweislich von Vorfahren mit dieser Gliederung abstammen.

Kulturelle Evolution -- Artentwicklung, die nicht auf genetischer Vererbung beruht, sondern auf Tradition. Für den Menschen wichtiger als biologische Evolution.

Kulturbedinqungen -- geeignetes Wachstumssubstrat (ECM); Temperatur und pH-Wert; Kulturmedium mit fötalem Kälberserum (undefiniert) oder geeigneten Wachstumsfaktoren(definiert).

Kinetochor -- Kinetochore sind spezialisierte Anheftungsstellen für Mikrotubuli am Centromer des Chromosoms

Kin Selection -- TODO

\end{minted}
\newpage


\section{L}
\begin{minted}[linenos,numberblanklines=false,tabsize=4,breaklines,breakindent=25pt]{haskell}
Lamarckismus -- die von Lamarck begründete Denkschule, wonach sich Lebewesen aufgrund eines Vervollkommnungstriebs anpassen und danach die erworbenen Eigenschaften weiter vererben

Latenzproblem -- Genetisches Problem, wie man erklärt, dass bei Kindern "verdeckte" Eigenschaften der Eltern auftreten. Wurde durch Mendel/Kölreuter über 2 Anlagen erklärt

Lipid Raft -- "Fettflösschen", Bereiche der Membran mit einem hohen Gehalt an \textbf{gesättigten} Fettsäuren (daher verdickt), wo zahlreiche Signalmoleküle (z.B. Rezeptoren) versammelt sind.

Lokus -- Spezifische Ort auf einen Chromosom wo man bestimmte Gene (Allele) findet

Lyssenkoismus -- sowjetische Spielart des Neolamarckismus, wo Genetik als "bourgeois" abgelehnt und durch "Weitergabe erworbener Eigenschaften" ersetzt wird, bis 1956 dominant

Lysosom :: bauen Dinge ab -- dienen der intrazellulären Verdauung, enthalten enzyme zum Abbau von Biopolymeren in Monomere, werden im u.a. im Golgi gebildet. Lysosomen entstehen aus speziellen Vesikeln des Golgi Apparats, die mit Endosomen fusionieren.

Lamin :: Intermediärfilament, Bestandteil der Kernlamina -- Lamine A, B und C sind Bestandteile der Kernlamina, polymerisieren zu einem 2-dimensionalen Netz, wird während der Zellteilung aufgelöst, stellt den Kontakt zum Chromatin her



\end{minted}
\newpage


\section{M}
\begin{minted}[linenos,numberblanklines=false,tabsize=4,breaklines,breakindent=25pt]{haskell}

Mayrischer Artbegriff -- 

Makroevolution -- Entstehung neuer Baupläne oder qualitativ neuer Strukturen, von Darwin komplett ausgeblendet

Meiose -- Sonderform der Mitose, wobei in der ersten Prophase Stücke von mütterlichen und väterlichen Chromosomen ausgetauscht werden und in der zweiten Teilung durch Ausfallen der DNS-Synthese die DNS-Menge halbiert wird. Kernstück der Sexualität

Mikroevolution -- Darwins Evolutionsmodell einer allmählichen Veränderungen in kleinen Schritten durch Variation und Selektion. Damit lässt sich nicht die Entstehung neuer Baupläne erklären.

Mikrotubuli -- rohrförmige Polymere aus Tubulin, an denen Kinesin und Dynein-Motoren entlanglaufen, Grundlage der Geißelbewegung, sowohl statischer als auch dynamischer Struktur, bestehen auch aus \alpha-Tubulin und \beta-Tubulin

Miller-Experiment -- Experimentelle Simulation der Situation auf der Urerde, dabei entstanden aus einfachen Molekülen komplexe Biomoleküle wie Aminosäuren, Nucleotide und Zucker

Mimikry -- Nachahmung von Organismen durch andere Organismen zum Zwecke der Tarnung

missing link -- zumeist hypothetische Zwischenformen eines evolutionären Wandels, die laut Darwin ein Mosaik aus "alt" und "neu" darstellen sollten. 

Mitochondrien :: ATP Produktion -- Zellorganellen, in denen die Atmung stattfindet, entstanden aus freilebenden Bakterien und haben daher eigene DNS und Ribosomen

Mitose -- Teilung des Zellkerns, in der Regel mit einer Teilung der ganzen Zelle (Cytokinese) verbunden.

Modularisierung -- Bauplan aller Lebewesen aus Elementen, die wiederholt und abgewandelt werden (Telome der Pflanzen, Segmente der Tiere, Gliedmaßen)

Musterbildung -- bei der Entwicklung die Herausbildung einer geordneten Anordung von Elementen

Mutation -- zufällige Veränderung der DNS-Sequenz, führt zumeist zu einer veränderten Abfolge der Aminosäuren im davon kodierten Protein

Myosin -- Motorprotein, das unter ATP-Spaltung entlang von Actin wandern kann. Grundlage der Muskelbewegung

Moonlighting Function -- TODO

Monoklonaler Antikörper -- erkennen alle die gleiche Domäne eines Antigens. Gegenteil: polyklonale Antikörper. Siehe auch: Experiment zur Herstellung von Köhler und Milstein

MPF-Kinase :: maturation promotion factor -- induziert die M-Phase in Zellen (Einleitung der Zellteilung). Die MPF-Aktivität ändert sich periodisch, sie steigt kurz vor Beginn der Mitose steil an und fällt gegen Ende der Mitose auf den Nullwert ab. Die Konzentration bleibt allerdings immer gleich. MPF phosphorlysiert Schlüsselproteine die u.a. das Cytosckelett umorganisieren und die Kernlamina abbauen. MPF besteht aus einer Cyclin und Cdk Untereinheit

\end{minted}
\newpage


\section{N}
\begin{minted}[linenos,numberblanklines=false,tabsize=4,breaklines,breakindent=25pt]{haskell}

Nacktsamer :: Gymnospermae. Unabhängigkeit vom Wasser bei der Fortpflanzung --  Entstanden in der Entwicklung nach Urfarnen. Was ist neu? Gametophyt (Gameten-bildende, sexuelle Generation, also die haploide Phase des Generationswechsels.) stark reduziert. Befruchtung mit Pollenschlauch. Embryo in Mutterpflanze

Nische -- der "Beruf" eines Organismus in einem Ökosystem, also die Überlebensstrategie, die verfolgt wird.

Nukleus :: Zellkern --

Nukleolus :: Ort der Ribosomenbiogenese -- Untereinheit des Nukleus, Im Nukleolus konzentrieren sich die transkriptionell aktiven Gene, die für die ribosomale RNA (rRNA) kodieren. Die rRNA assoziiert im NuKleolus mit den im Zytoplasma synthetisierten ribosomalen Proteinen zu den ribosomalen Untereinheiten. Die ribosomalen Untereinheiten werden dann ins Cytosol exportiert

Nukleosom :: komplex aus Histonen und DNA

Nekrotrophie -- Parasit tötet die Zellen des Wirts und saugt sie aus. Typische für eine evolutionär junge Situation. Gegenteil: Biotrophie


\end{minted}
\newpage


\section{O}
\begin{minted}[linenos,numberblanklines=false,tabsize=4,breaklines,breakindent=25pt]{haskell}

Oberflächen-Komplementarität -- von Proteinen ermöglicht die spezifische Interaktion mit anderen Proteinen über nicht-kovalente Wechselwirkungen.

ökologische Potenz -- Bandbreite eines Organismus, die aufgrund seiner Physiologie und der Konkurrenz mit anderen Arten möglich ist.

Onkogen -- Krebsförderndes Ges. Oft aus einem Proto-Onkogen hervorgegagen (proliferationsförderndes Gen)

Operon -- Einheit aus mehreren Genen, die gemeinsam für eine Funktion notwendig sind und durch denselben Promotor gesteuert werden.

Osmose -- Volumenänderungen von membranumschlossenen Räumen (Zellen), weil Wasser durch die Membran kann, Salze oder Zucker aber nicht.

Okazagi Fragment -- DNA Fragmente von Primer bis Primer auf dem lagging strand (Folgestrang) beid er DNA Replikation
\end{minted}
\newpage


\section{P}
\begin{minted}[linenos,numberblanklines=false,tabsize=4,breaklines,breakindent=25pt]{haskell}
Panspermie-Hypothese -- Die Hypothese, dass das Leben über Meteoriten im Weltall verbreitet wurde. Löst die Frage nach der Entstehung von Leben nicht. 

Parazoa -- Schein-Vielzeller, Kolonien aus Einzelzellen, die nicht wirklich ein Ganzes bilden (z.B. Schwämme)

PCR -- Polymerase Ketten (chain) Reaktion. Wichtiges Verfahren der Molekularbiologie, bei dem bestimmte DNS-Stücke mithilfe von hybridisierenden kurzen Erkennungssequenzen (Oligonucleotidprimern) vervielfacht werden. Man benutzt dazu hitzestabile Polymerasen, die das DNS-Stück zwischen den Erkennungssequenzen verdoppeln. Danach wird die gebildete DNS neu aufgeschmolzen und der Schritt wiederholt.

Phenotyp -- Erscheinungsform eines Merkmales

Phloëm -- sprich Phlo-em, Leitelement der Kormophyten, das Assimilate von den Blättern zu den Verbrauchsorganen bringt.

Phospholipide -- Bausteine der Membran, bestehen aus einem Glycerolrest, wo zwei OH-Gruppen mit Fettsäuren, die dritte OH-Gruppe über eine Phosphatgruppe mit einem weiteren Molekül verestert ist.

Photosysteme -- Komplexe aus Chlorophyll, Proteinen und Antennenpigmenten (Carotinoiden) in der Thylakoidmembran, wo die Lichtreaktionen der Photosynthese stattfindet. Viele Chlorophylle sind zu Lichtsammelkomplexen vereinigt, wobei nach Anregung ein Strom ans Reaktionszentrum des Photosystems fließt und von dort die weiteren Membranvorgänge gestartet werden.

physiologische Potenz -- Bandbreite eines Organismus, die aufgrund seiner Physiologie möglich ist

Plankton -- frei schwimmende Mikroorganismen (Algen, Protisten, kleine Tiere)

Pleiotropie -- Die Befähigung eines Gens, den Phänotyp eines Organismus in \textbf{vielfacher} Weiße zu beeinflussen

Plasmodesmen -- plasmatische Verbindung pflanzlicher Zelle, mikroskopische sichtbar als "Tüpfel".

Plasmolyse -- Sonderfall der Osmose, wobei Zellen in einem hypertonischen Medium (Zucker- oder Salzlösung) Wasser verlieren und schrumpfen

Plastiden -- semiautonome Organellen pflanzlicher Zellen mit eigener DNS und eigenen Ribosomen, die für die Photosynthese (Chloroplasten) verantwortlich sind, aber auch noch andere Stoffwechselwege (Carotinoidsynthese, Stärkebildung) beherbergen. Diese anderen Stoffwechselwege können gewebsabhängig hochgeregelt sein, wodurch Chromoplasten und Amyloplasten entstehen. Die Plastiden leiten sich von prokaryotischen Cyanobakterien ab.

Polarität -- eine "Richtung" von Faktoren, die Entwicklung steuern

Polyklonale Antikörper -- verschiedene Antikörper, die unterschiedliche Domänen eines Antigens erkennen. Gegenteil: monoklonale Antikörper

Polygenismus -- Idee u.a. von Agassiz, dass die Menschenrassen nicht verwandt, sondern unabhängig geschaffen worden seien (auf der Basis der Artkonstanz)

Präadaptation -- makroevolutionäres Konzept, wonach ein Merkmal schon vorbereitet wird, obwohl die Notwendigkeit erst später kommen wird. Wird inzwischen durch Funktionswechsel erklärt.

Präformation -- Modell der Entwicklung, wo die Gestalt des Organismus schon vorgeformt im Ei vorliegt

primäre Atmosphäre -- reduzierende (sauerstoff-freie) Atmosphäre der Urerde

Primärstruktur (Protein) -- lineare Anordnung oder Sequenz der Aminosäuren in einer Polypeptidgruppe

Produzenten -- in ökologischen Systemen die Organismen, die photosynthetisch Energie binden

Primärkultur -- Tiereische Zellen können enzymatisch aus dem Gewebeverband gelöst werden und unter geeigeneten Bediungen einige Zeit in vitro gehalten werden

Prokaryoten -- Organismen, bei denen die DNS frei vorliegt, haben also keinen Zellkern - Bakterien und Cyanobakterien

Promiskuität -- Geschlechtsverkehr mit häufig wechselnden Partnern

Proteasom :: Fleischwolf für falsch gefaltete Proteine -- Fehlgefaltete bzw. überalterte cytoplasmatische Proteine werden durch kovalente Anbindung von mehreren kleinen Proteinen (Ubiquitin) markiert, dann in den Tunnelartigen Hohlraum des Proteasoms (hochmolekularer Proteinkomplex mit enzymatischer Funktion) eingeschleust und dort in kleine Peptide zerlegt.

Protostomier -- Tiergruppe, bei der der Urmund der Gastrula zum Mund wird - die "Wirbellosen"

Proteinmotive -- wiederkehrende Kombinationen von Aminosäuren die mit spezifischen Funktionen assoziiert sind.

Proteindomäne -- einen Bereichs des Proteins mit definierter Faltungsstruktur, der funktionell und strukturell unabhängig von benachbarten Abschnitten ist.

Polymerase -- Polymerasen sind in allen Lebewesen vorkommende Enzyme, die die Polymerisation von Nukleotiden, die Grundbausteine der Nukleinsäure, katalysieren.

Pathogen -- Pathogene (Substantiv) sind Mikroorganismen, Viren, Gifte und ionisierende Strahlung, die eine Erkrankung hervorrufen können. Es handelt sich dann um pathogene (Adjektiv) Erreger bzw. Substanzen.

Proteasom -- Das Proteasom (auch: Macropain) ist ein Proteinkomplex von 1.700 kDa, der im Cytoplasma und im Zellkern (bei Eukaryoten) Proteine zu Fragmenten abbaut und daher zu den Peptidasen (auch Proteasen) zählt. Das Proteasom ist ein Bestandteil der Proteinqualitätskontrolle.

Phagocytose -- Phagocytose (von phagein ‚fressen'), eingedeutscht auch Phagozytose, bezeichnet die aktive Aufnahme von Partikeln (bis zu kleineren Zellen) in eine einzelne eukaryotische Zelle.

Phospholipase C :: produziert IP3 und DAG --

Peroxisom :: bauen zellgifte ab, u.a. H2O2 -- membranumhüllte Organellen mit peroxidativ aktiven Enzymen. Funktion ist u.a. H2O2 abzubauen.

Polyploidie -- Verfielfachung von Chromosomen oder ganzen Chromosomensätzen

PQ Formel -- Hardy Weinberg Formel, zur Berechnung von Häufigkeiten in Vererbungen. 2pq ist wie viele Leute Aa tragen, p^2 AA, q^2 aa

Parakrin :: wie endokrin, aber wirkt nur auf umbegende Zellen -- Als parakrin bezeichnet man den Sekretionsmodus von innersekretorischen Drüsenzellen, ihre Produkte in das Interstitium 665901ihrer unmittelbaren Umgebung abzugeben. Parakrin ist daher ein ergänzender und in manchen Anwendungen auch gegensätzlicher Begriff zu endokrin. 

Proteinkinase --  werden nach Aktivierung von Zelloberflächenrezeptoren aktiviert und phospholysieren dann Proteine -> Intrazelluläres Signalprotein. Gegensatz: Proteinphosphatasen dephospholysieren Proteine

Proteinphosphatase -- dephospholysieren Proteine

Punctuate Equilibrium -- Theorie. Keine langsame und mit konstanter Geschwindigkeit fortschreitende Transformation biologischer Arten, sondern Abwechslung von "Stasis" (Stillstand) (Arten haben geringes Ausmaß an morphologisch auffälliger Veränderung), und schnellem Wandel während der allopatrischen Artbildung -> "Stasis" wird durchbrochen (engl. punctuated).665901

Plasmid -- ringförmige DNA bei Prokarioten (Bakterien)

\end{minted}

PQ Formel: $p^{2}$ + $2pq$ + $q^2$ = 1, $p + q = 1$, p Häufigkeit von A, q Häufigkeit von a 

\newpage


\section{Q}
\begin{minted}[linenos,numberblanklines=false,tabsize=4,breaklines,breakindent=25pt]{haskell}

Quartärstruktur (Protein) -- Manche Proteine bestehen aus Multimeren. Die Quartärstruktur beschreibt die Anzahl und die räumliche Anordnung der Untereinheiten

\end{minted}
\newpage


\section{R}
\begin{minted}[linenos,numberblanklines=false,tabsize=4,breaklines,breakindent=25pt]{haskell}
r-Strategie -- ökologische Strategie von Organismen, die auf schnelle Vermehrung und Überdauerung abzielt

Rekombination -- die Neukombination von Genen bei der Meiose

Rekombinationsfrequenz -- (Anz Rekombinationen / Anz Nachkommen insgesammt) * 100 

Replikation -- Verdopplung der DNS

Rezeptortyrosinkinasen -- besitzen nur eine Transmembrandomäne. Die Bindung eines Liganden kann daher nicht zu einen Konformationsänderung in der cytosolischen Domäne führen. Daher dimerisieren diese Rezeptoren nach Ligandenbindung, was zu einer gegenseitigen Phosphorylierung führt. Viele Rezeptoren für Wachstumsfaktoren gehören in die Klasse der Rezeptortyrosinkinasen

Restriktionsenzyme -- Werkzeug der Molekularbiologie, Enzyme, die DNS gezielt an bestimmten Erkennungsstellen schneiden

Rhinogradentier -- in der Karlsruher Biologie entdeckte Organismengruppe, die über adaptive Radiation ein pazifisches Archipel besiedelte, leider ausgestorben.

Ribosom -- Organell, an dem im Cytoplasma die Translation stattfindet, besteht aus Protein und rRNS

RNS-Welt -- Hypothese, dass vor der DNS-Protein Welt von heute Lebensformen mit RNS als Erb- und Enyzmsubstanz vorherrschten. Die RNS-Natur von rRNS, mRNS und tRNS gilt als wichtiger Beleg für dieses Modell.

rezessiv :: unterordnend, Gegenteil von Dominant -- rezessiv vererbt heißt beide allele müssen tragen

Ribosomenbiogenese :: Ribsosomen herstellen -- 

RubisCo -- Fixierung von Kohlendioxid an Zucker durch das Enzym Ribulose-bisphosphat-Carboxylase (RubisCo). Wenig CO2 und viel O2 (Stomata geschlossen): RubisCo wirkt in Gegenrichtung als Decarboxylase (Photorespiration, giftig). Enstand vor langer Zeit, als es noch keinen sauerstoff gab

R-Plasmide -- Übertragen Resistenzgene gegen bestimmte Antibiotica oder Schwermetalle. Nicht zu verwechseln mit F-Plasmid

\end{minted}
\newpage


\section{S}
\begin{minted}[linenos,numberblanklines=false,tabsize=4,breaklines,breakindent=25pt]{haskell}
sekundäre Atmosphäre -- die oxidative Atmosphäre, wie wir sie heute haben, die aus der reduzierenden primären Atmosphäre infolge der Photosynthese entstanden ist

Sekundärstruktur (Protein) -- Faltung oder räumliche Anordnung oder Polypeptidkette durch Wechselwirkungen benachbarter Aminosäuren (ionische Bindungen, Wasserstoffbrücken, Van der Waals-WW). Es ergeben sich: Alpha-Helix, Beta-Faltblatt, Zufallknäuel, Haarnadelschleifen

Sarkomer -- kleinste funktionelle Einheit der Muskelfibrille (Myofibrille) und somit der Muskulatur.

Selektion -- "Zuchtwahl", Auslese der Individuen, die sich fortpflanzen. Bei der Züchtung sucht der Züchter aus, wer sich fortpflanzt. Bei der "natürlichen Zuchtwahl" entscheiden die Umweltbedingungen, welche Individuen übrigbleiben und sich fortpflanzen.

Semipermeabilität -- Halbdurchlässigkeit von Membranen, Wasser und kleine apolare Moleküle gelangen hindurch, große Moleküle und Ionen nicht

sexuelle Selektion -- Auswahl des Geschlechtspartners durch das andere Geschlecht aufgrund von besonderen Merkmalen, die etwas über die fitness aussagen und zu einem sexuellen Dimorphismus führen.

Soma -- bei der Entwicklung entstehende Zell-Linien, die nicht an der Fortpflanzung teilnehmen und daher sterblich sind.

Sozialdarwinismus -- Übertragung darwinistischer Evolution auf die menschliche Gesellschaft

Spaltungsregel -- auch 2. Mendelgesetz: bei der Selbstung von Heterozygoten (Aa) spaltet sich die Nachkommenschaft nach 1:2:1 in AA, Aa und aa auf.

Speziation -- Artbildung, zumeist allopatrisch (nach geographischer Auftrennung einer Population), oder sympatrisch (keine geogr. Abtrennung)

Substrataktivierung -- Fall der Gensteuerung, bei dem ein Operon durch den (umzusetzenden) Ausgangsstoff angeschaltet wird. Beispiel: Lactose-Abbau

sympatrisch -- Variante der Artbildung, bei der die Subpopulation ohne vorausgehende geographische Trennung eine Fortpflanzungsbarriere ausbildet. Gegenteil: allopatrisch

synaptonemaler Komplex -- molekularer Reißverschluss, der mütterliche und väterliche Chromosomen passgenau nebeneinander legt - Voraussetzung für die Meiose

Synthetische Theorie -- Erweiterung von Darwins Theorie um populationsgenetische (Speziation), entwicklungsbiologische (Embryologie) und zellbiologische (Endosymbiose) Elemente.

second Messenger :: kleine intrazelluläre Signalmoleküle -- Nach der Bindung der Liganden wird in vielen Fällen die Konzentration an kleinen (schnelle Diffusion!), intrazellulären Signalstoffen (second messenger) verändert. cyclisches Adenosin-Monophosphat (cAMP), cyclisches Guanosin-Monophosphat (cGMP), Diacylglycerin (DAG), Inositol-Triphosphat (IP3), Verschiedene Phosphoinositide, Kalzium-Ionen (Ca2+)

Signaldistanzen :: Endokrin, Parakrin, Autokrin, Synaptisch, Kontaktabhängig

\end{minted}
\newpage


\section{T}
\begin{minted}[linenos,numberblanklines=false,tabsize=4,breaklines,breakindent=25pt]{haskell}
Telom -- Baumodul aller höheren Landpflanzen (Kormophyten) bestehend aus Leitbündel, Parenchym und Epidermis

Tertiärstruktur (Protein) -- Gesamtkonformation einer Polypeptidkette in ihrer dreidimensionalen Anordung aller Aminosäuren. Entsteht durch Interaktionen aller Sekundärstrukturen, wird meistens durch hydrophone WW zwischen apolaren Seitenketten stabilisiert 

Thylakoidmembran -- innere Membran des Chloroplasten, zu Stapeln (Grana) aufgeschichtet. Hier findet die Lichtreaktion der Photosynthese statt.

Transition -- Mutation, bei der der Basentyp (Purin oder Pyrimidin) beibehalten wird

Transit Amplifying Cells -- teilen sich symmetrisch. Sie vergrößern die Nachkommenschaft einer Stammzelle ohne ihr Teilungspotenzial zu strapazieren.

Transkription -- Ablesen der mRNS von der DNS (passiert im Zellkern)

Translation -- Übersetzen der mRNS in Protein (passiert im Cytoplasma an den Ribosomen)

Transversion -- Mutation, bei der eine Purin- in eine Pyrimidinbase umgewandelt wird (oder umgekehrt).

Turgor -- Wanddruck der Pflanzenzellen, entsteht durch Osmose, wodurch sich die Zelle ausdehnt und auf die Wand einen Druck ausübt, der das Wachstum antreibt.

Telomer -- spezialisierte Struktur an den den Enden der Chromosomen, sichern die komplette Replikation, schützen Enden vor enzymatischem Abbau

Telomerase -- Die Telomerase verhindert durch die Wiederherstellung der Telomere, dass die Chromosomen mit jeder Zellteilung kürzer werden, was schließlich zum Zelltod führen würde. Die Telomerase ist ein Enzym des Zellkerns, welches aus einem Protein- (TERT) und einem langen RNA-Anteil (TR) besteht und somit ein Ribonukleoprotein ist. Dieses Enzym stellt die Endstücke der Chromosomen, die sogenannten Telomere, wieder her.

Transmembranprotein :: stecken in der zellmembran und haben eine verbindung ins cytoplasma und das äußere -- 

Treadmilling -- neu angebautes Aktin wird in Gleichgewichtsphase der Aktinfilamente zum anderen Ende durchgereicht und dort abgebaut

Transformation -- Genaustauschs zwischen Bakterien

\end{minted}
\newpage


\section{U}
\begin{minted}[linenos,numberblanklines=false,tabsize=4,breaklines,breakindent=25pt]{haskell}
Uniformitätsregel -- auch 1. Mendelgesetz: bei der Kreuzung zweier homozygoter Eltern (AA mit aa) entstehen einheitliche Nachkommen (Aa)

Unvollständige Dominanz -- Dominante Homozygote zeigen einen stärker ausgeprägten Phänotyp als die Heterozygoten (intermediärer Erbgang). Das Gegenteil: vollständige Dominanz

Ursuppentheorie :: mittlerweile stark angezweifelt -- biologisch relevante, organische Verbindungen enstanden durch chemische Prozesse in der Atmosphäre und reicherten sich in den Weltmeeren an. Dieser Ursupee enstiegen im Laufe der Zeit komplexere Biosysteme

\end{minted}
\newpage


\section{V}
\begin{minted}[linenos,numberblanklines=false,tabsize=4,breaklines,breakindent=25pt]{haskell}
Variation - die genetische Unterschiedlichkeit von Individuen einer Population, Voraussetzung für die Evolution

Vektoren -- ringförmige DNS-Stücke von Bakterien, auf denen einzelne Gene für bestimmte Eigenschaften weitergegeben werden können. Diese wurden für molekularbiologische Zwecke umgebaut, um Gene von Interesse vervielfältigen und auf andere Organismen übertragen zu können.

Vollständige Dominanz -- Die Phänotypen der Heterozygoten und dominanten Homozygoten Individuen zeigen keinen Unterschied. Das Gegenteil: unvollständige Dominanz

\end{minted}
\newpage


\section{W}
\begin{minted}[linenos,numberblanklines=false,tabsize=4,breaklines,breakindent=25pt]{haskell}

\end{minted}
\newpage


\section{X}
\begin{minted}[linenos,numberblanklines=false,tabsize=4,breaklines,breakindent=25pt]{haskell}
Xylem -- Wasserleitungsgewebe der Kormophyten, verholzt, transportiert Wasser und Mineralien von der Wurzel nach oben

\end{minted}
\newpage


\section{Y}
\begin{minted}[linenos,numberblanklines=false,tabsize=4,breaklines,breakindent=25pt]{haskell}

\end{minted}
\newpage


\section{Z}
\begin{minted}[linenos,numberblanklines=false,tabsize=4,breaklines,breakindent=25pt]{haskell}
Zuchtwahl -- alter Begriff für Selektion, auf Darwins Modell des Taubenzüchters beruhend

Zelllinien -- Primärkulturen sterben nach einer bestimmten Anzahl von Teilungen ab. Zelllinien werden entweder aus Tumoren isoliert, entstehen spontan in Kultur, oder werden durch Transformation mit z.B. Viren hergestellt.

\end{minted}
\newpage


\section{Proteine}

\begin{itemize}

    \item Myosin
        \begin{itemize}
            \item aktinabhängige Motorproteine und bilden eine große Proteinfamilie. Sie binden und hydrolysieren ATP; diese Energie wird benutzt um entlang des Aktinfilaments vom Minus- zum Plus-Ende zu wandern. Myosine vom Typ-I setzen sich aus einer einzelnen Kopf- und einer Schwanzdomäne zusammen und kommen in allen Zellen vor. Myosine vom Typ-II besitzen zwei Kopfdomänen und sind spezifisch für Muskelzellen.
        \end{itemize}

    \item Tropomysosin    
        \begin{itemize}
            \item blockiert die Myosin-Bindungsstelle auf dem Aktinfilament. Wenn Ca2+ dazukommt, gibt Tropomysosin die die Bindungsstelle für Myosin auf dem Aktinfilamnent im Muskel frei und der Muskel kontrahiert.
        \end{itemize}
    
    \item Cap Z
        \begin{itemize}
            \item blockiert die Myosin-Bindungsstelle auf dem Aktinfilament am Plus-Ende; verhindert eine Dissoziation
        \end{itemize}
        
    \item Nebulin \& Titin
        \begin{itemize}
            \item sind in der Z-Scheibe verankert und tragen zur passiven Spannung des Muskels bei
        \end{itemize}
        
    \item Tropomodulin
        \begin{itemize}
            \item blockiert die Myosin-Bindungsstelle auf dem Aktinfilament am Minus-Ende; verhindert eine Dissoziation
        \end{itemize}
    
    \item Arp2/3
        \begin{itemize}
            \item bindet an das Minus-Ende von Aktin (Nukleationskeim) 
        \end{itemize}
        
    \item Dynein
        \begin{itemize}
            \item wandert entlang Mikrotubuli Richtung Minus-Ende
        \end{itemize}
        
    \item Kynesin
        \begin{itemize}
            \item wandert entlang Mikrotubuli Richtung Plus-Ende. Befördern Vesikel. Besitzt eine dimere Kopf- domäne, welche sich an Mikrotubuli und ATP bindet, sowie eine Schwanz- domäne, die mit Vesikeln in Wechsel- wirkung tritt.
        \end{itemize}
    
    \item Taxol, Colchizin, Vinblastin    
        \begin{itemize}
            \item Pflanzliche Alkaloide beeinflussen die Polymerisation und Stabilität von Mikrotubuli. Sie finden daher therapeutischen Einsatz in der Chemotherapie, weil sie die Funktion der Kernteilungsspindel beeinträchtigen. Stabilisierend: Taxol (Eibe). Destabilisierend: Colchizin (Herbstzeitlose), Vinblastin (Immergrün)
        \end{itemize}

    \item Cytochalasin
        \begin{itemize}
            \item (aus primitiven Pilzen) binden an das Plus- Ende der Aktin-Filamente und verhindern eine weitere Polymerisation.
        \end{itemize}
        
    \item Phalloidin
        \begin{itemize}
            \item (aus Knollenblätterpilz) bindet an Aktinfilamente und stabilisiert sie. Daher der Einsatz zur Markierung von Aktin.
        \end{itemize}
        
    \item Latrunculin
        \begin{itemize}
            \item (aus Seeschwamm) bindet an Aktinmonomere und verhindert so die Polymerisation.
        \end{itemize}

    \item G-Proteine
        \begin{itemize}
            \item große intrazelluläre Signalmoleküle
            \item wie ein Schalter, der Signale von ausserhalb der Zelle nach innerhalb weiterleitet
            \item bestehen aus 3 Proteinuntereinheiten.Im Ruhezustand ist GDP an die a- Untereinheit gebunden. Nach Bindung eines Liganden an den Rezeptor bindet das G-Protein und es wird GDP durch GTP ausgetauscht. Das G-Protein zerfällt in 2 aktive Untereinheiten (a und ßg-Komplex), die beide mit Zielstrukturen in der Membran wechselwirken können. \textbf{Die häufigsten Zielproteine sind: Ionenkanäle, Adenylatcyclase (stellt cAMP her), Phospholipase C (produziert IP3 und DAG)} 
        \end{itemize}
        
    \item G-Protein-gekoppelte Rezeptoren
        \begin{itemize}
            \item besitzen 7 Transmembran-domänen (N- Terminus extrazellulär, C-Terminus intrazellulär). Nach Bindung eines Liganden verändert sich die intrazelluläre Konformation des Rezeptors und das trimere G-Protein kann binden. Zu dieser Familie gehören lichtaktivierte Rezeptoren (Rhodopsin), Riechstoff-Rezeptoren, Rezeptoren für Hormone (Adrenalin, Noradrenalin) und für Neurotransmitter.
            \item Ras ist auch ein G-Protein
        \end{itemize}

    \item Ras-Protein
        \begin{itemize}
            \item Ras gehört zu einer Familie von kleinen GTP-bindenden Proteinen. Diese sind an vielen Prozessen der Signaltransduktion von Wachstumsfaktoren beteiligt.
            \item \textbf{gehört zu den G-Proteinen}
        \end{itemize}

    \item Rhodopsin
        \begin{itemize}
            \item \textit{siehe G-Protein-gekoppelte Rezeptoren}
        \end{itemize}
 
    \item MAP
        \begin{itemize}
            \item stabilisiert Mikrotubuli -> längere, weniger dynamische Mikrotubuli
        \end{itemize}
    
    \item p53
        \begin{itemize}
            \item wird nach DNA-Schädigung aktiviert, Stimuliert dann die Expression von p21, einem Cdk-Inhibitor
            \item p21 bindet an den Cyclin-Cdk-Komplex (MPF) und inaktiviert ihn
            \item dadurch hält der Zellzyklus in der G1 Phase an
        \end{itemize}
        
    \item Cdk
        \begin{itemize}
            \item cyclinabhängige Proteinkinase
            \item Teil des MPF Komplex
        \end{itemize}
        
    Cyclin -- Protein, das in einem Komplex mit einer Kinase den Zellzyklus weiterführt. Nach dem Zellzyklusschritt wird es abgebaut, um eine Dauerteilung zu. verhindern. Cyclin bildet einen Komplex mit MPF 
    
    \item Cyclin
        \begin{itemize}
            \item zyklischer Anstieg und Abfall der Konzentration wöhrend des Zellzyklus
            \item in einem Komplex mit einer Kinase (MPF-Komplex)
        \end{itemize}
        
    \item Catastrophin
        \begin{itemize}
            \item erhöht die Frequenz des Abbaus der Mikrotubuli -> kürzere, dynamischere Mikrotubuli
        \end{itemize}

    \item Kinesin
        \begin{itemize}
            \item 
        \end{itemize}
        
    \item Profilin
        \begin{itemize}
            \item fördert Wachstum von Aktinfilamenten
        \end{itemize}        
        
    \item Thymosin
        \begin{itemize}
            \item hemmt Wachstum von Aktinfilamenten
        \end{itemize}        

    \item Gelsolin
        \begin{itemize}
            \item schneidet Aktinfilament an bestimmter Stelle
            \item verhindert anschließend Anbindung neuer Aktine an das Ende $\Rightarrow$ Capping
            \item erhöht die Dynamik des Filamentaufbaus
        \end{itemize}
        
    \item cAMP
        \begin{itemize}
            \item Cyclisches Adenosinmonophosphat (cAMP) ist ein vom Adenosintriphosphat (ATP) abgeleitetes biologisches Molekül, welches als Second Messenger bei der zellulären Signaltransduktion dient und insbesondere zur Aktivierung von Proteinkinasen führt
        \end{itemize}
        
    \item TODO: alle Proteine die an der DNA Replikation beteiligt sind (Helikase, DNA-Ligase, ...)
        \begin{itemize}
            \item 
        \end{itemize}
        
    \item Rho-Kinase
        \begin{itemize}
            \item ist so an der Regulation verschiedener Zellfunktionen beteiligt, wie der Kontraktion glatter Muskelzellen, der Organisation des Aktin-Zytoskeletts, Zellwanderung
        \end{itemize}  
        
    \item HSP60-like protein complex -- falsch gefaltete Proteine kommen da rein (wie so ein Fass mit Deckel) und werden unter ATP-Verbrauch korrekt gefaltet
        \begin{itemize}
            \item 
        \end{itemize}  
        
\end{itemize}

\newpage

\section{Experimente}

\begin{itemize}

    \item Mendel
        \begin{itemize}
            \item Nach welchen Regeln werden bestimmte Merkmale von Eltern (Parentalgeneration) auf die Nachkommen (Filialgeneration) weitergegeben? -> Genetische Kreuzung verschiedener Sorten der Gartenerbse und quantitative Bestimmung der untersuchten Merkmale bei den Nachkommen
        \end{itemize}        
        
    \item Entdeckung genetischer Kopplung durch Thomas Hunt Morgan 
        \begin{itemize}
            \item genetische Experimente mit Fruchtfliegen (kurze Generationszeit, nur 4 Chromosomenpaare)
            \item kreuzt grau, normalflügelig heterozygot mit schwarz, stummelflügelig homozygot
            \item Die Nachkommensanzahl mit den jeweiligen Merkmalen ist anders als erwartet
            \item Das liegt daran dass die Gene auf dem selben Chromosom liegen
        \end{itemize}   

    \item Avery, McCarty and McLeod weisen DNA als genetisches Material nach (1944)
        \begin{itemize}
            \item konnten die transformierende Substanz aus F. Griffiths Experiment isolieren und in einem Reagenzglas bereitstellen. Hinzufügen verschiedener Enzyme: Trypsin (Abbau von Proteinen), Ribonuclease (Abbau von RNA), SIII Enzym (Abbau von Polysaccharide) änderten nichts an den transformierenden Eigenschaften der Substanz. Hinzufügen von Desoxyribonuclease (Abbau von DNA): Keine Transformation mehr! Cheminsche Zusammensetzung der transformierenden Substanz gleicht der von DNA -> muss DNA sein
            \item aber niemand glaubte ihm :(
        \end{itemize} 

    \item Hershey-Chase-Experiment (1952)
        \begin{itemize}
            \item Infizierendes Material von Bakterien-befallenden Viren (Bakterio- phagen) besteht aus DNA und nicht aus Proteinen. Nachweis durch radioaktive Markierung von Phagen-Proteinen mit radioaktivem Schwefel und Phagen-DNA mit radioaktivem Phosphor
            \item zwei parallele Experimente: einmal DNA radioaktiv markiert, einmal Protein radioaktiv markiert
            \item radioaktiv markierte Phagen infizieren Bakterien -> geben genetisches Material in Bakterien ab
            \item starkes Rühren entfernt Phagen von Bakterien
            \item zentrifugieren der Mischung
            \item wenn bei Proteinexperiment Radioaktivität in Bakertium -> Protein muss genetisches Material sein
            \item wenn bei DNA-Enexperiment Radioaktivität in Bakertium -> DNA muss genetisches Material sein
            \item Beim Proteinexperiment: Radioaktivität in den Phagen, ausserhalb des Bakteriums
            \item Beim DNA-Experiment: Radioaktivität in den Bakterien, nicht mehr in den Phagen
            \item -> genetisches Material MUSS DNA sein
        \end{itemize}        
        
    \item F. Griffith entdeckt Transformation (Genaustauschs zwischen Bakterien) bei Bakterien (1928)
        \begin{itemize}
            \item Griffith arbeitete mit zwei Stämmen von Pneumokokken, dem virulenten S-Stamm, welcher über eine schützende Schleimkapsel verfügt, die der Bakterienkolonie ein glattes, glänzendes Aussehen verleiht und die deshalb smooth (S) genannt wurde, sowie dem nonvirulenten R-Stamm (R36A), Bakterien ohne Schleimkapsel und daher mit rauer Oberfläche, die rough (R) bezeichnet wurden. Griffith injizierte drei Gruppen von Mäusen unterschiedliche Extrakte: der ersten eine lebendige R-Stamm-Kultur, der zweiten durch Hitze getöteten S-Pneumokokken und der dritten beide Extrakte zusammen. Die erste und zweite Gruppe erkrankten nicht an Lungenentzündung. Die Mäuse der dritten Gruppe aber erkrankten und starben. Eine Kultur des Herzblutes dieser Mäuse zeigte wieder lebendige S-Stamm-Pneumokokken. Dadurch nahm Griffith an, dass die abgetöteten S-Pneumokokken eine transformierende Substanz enthielten, die den R-Typ in den S-Typ umwandeln kann.
        \end{itemize}        
        
    \item Meselson-Stahl-Experiment 1958
        \begin{itemize}
            \item Konnte semikonservative DNA Replikation beweisen. DNA mit modifizierten Basen (N15 (schwerem Stickstoff) anstatt N14) wurde erzeugt, Bakertien damit wurden in N14 medium transferiert, nach mehreren Zellteilungen konnte man die DNA zentrifugieren. Vergleich: zentrifugieren vor dem Transfer in N14 Medium. Ergebnisse: Vor dem Transfer war die DNA natürlich nur ganz weit aussen bei der Zentrifuge, weil N15 schwer ist. Nach dem Transfer und EINER Zellteilungen hatte man nur DNA in der Mitte (Gemisch von N14 und N15 50/50), nach ZWEI Zellteilungen hatte man DNA ganz oben (nur N14) und DNA in der Mitte (Gemisch aus N14/N15).
        \end{itemize}                
        
    \item Technnik zur Herstellung von monoklonalen Antikörpern von Köhler und Milstein 1975
        \begin{itemize}
            \item Maus wird Antigen X injiziert
            \item B-Zellen werden angeregt und produzieren Antikörper
            \item B-Zellen finden sich am meisten in der Milz, man extrahiert daher Milzzellen der Maus
            \item darunter befinden sich dann u.A. auch die B-Zellen die Antikörper gegen X produzieren
            \item mutierte Krebszellen teilen sich ewig, überleben aber nicht im HAT-Medium
            \item wenn man diese Krebszellen zusammen mit den Milzzellen fusioniert, werden die Milzzellen unsterblich, überleben aber weiterhin im HAT-Medium
            \item diese B-Zellen vermehren sich jetzt "für immer und ewig" -> monoklonaler Ursprung
            \item die B-Zellen produzieren weiterhin Antikörper gegen X -> kann man nun extrahieren
        \end{itemize}        

    \item DNA-Strukturaufklärung durch Röntgenbeugung (Rosalind Frankling und Maurice Wilkins 1952)
        \begin{itemize}
            \item reine DNA kristallisieren, dann röntgen
            \item Röntgenbeugungsmuster zeigt Periodizität 3,4 nm und 0,34 nm
            \item -> Abstand der Basen übereinander & Periodizität der Doppelhelixstruktur
        \end{itemize}        

    \item Fluktuationstest von Luria und Delbrück 1943
        \begin{itemize}
            \item um herauszufinden ob Mutationen (Phagenresistenz) durch Phagen induziert werden, oder zufällt sowieso auftreten 
            \item wenn induziert: in jeder Kultur tritt ähnliche Anzahl von resistenten Bakterien auf
            \item wenn zufällig: starke Fluktuationen in der Anzahl der resistenten Kulturen
            \item Bakterienlösung (nicht resistent gegen Phagen) wird verdünnt und mehrere Separate Kulturen gebildet
            \item warten bis in jeder Kolonie ca 1 Mrd Bakterien
            \item Jede Kultur wird auf Nährplatte mit Bakteriophagen gebracht (sind tödlich für alle nichtresistenten Bakterien)
            \item nach Wartezeit wird die Anzahl der resistenten Kolonien gezählt
            \item Ergebnis: zufällige Mutationen, nicht induziert
        \end{itemize}        
        
    \item U-Rohr von Bernard Davis
        \begin{itemize}
            \item beweist Notwendigkeit des physikalischen Kontakts zwischen F- und F+ Zelle für die Konjugation
            \item U-Rohr mit Membran in der Mitte, in jeweils einer Hälfte des Rohrs sind F+ und F- Bakterienstämme in einem Medium
            \item das Medium wird hin und her bewegt, die Zellen können aber nicht die Membran passieren
            \item keine Gene werden ausgetauscht, beide Bakterienstämme überleben nicht auf Mininmalmedium
        \end{itemize}        
        
    \item Experiment von Lederberg und Zinner 1952
        \begin{itemize}
            \item ähnlicher Aufbau mit U-Rohr wie von Bernard Davis
            \item Filter in der Mitte der Bakteriophagen durchlässt, Bakterien nicht
            \item rechts und links verschiedene Bakterienstämme
            \item linke Kolonie stirbt in Minimalmedium
            \item rechte Kolonie überlebt in Minimalmedium, weil es durch Phagen die gene aus der linken Kolonie bekommen hat die der rechten gefehlt haben
            \item DNase (baut DNA ab) ändert Resultat nicht -> Transformation kann ausgeschlossen werden
        \end{itemize}  
        
    \item Miller-Experiment 1952
        \begin{itemize}
            \item Experimentelle Simulation der Situation auf der Urerde, dabei entstanden aus einfachen Molekülen komplexe Biomoleküle wie Aminosäuren, Nucleotide und Zucker
        \end{itemize}
        
\end{itemize}

\end{document}