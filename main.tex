\documentclass{article}
\usepackage[utf8]{inputenc}
\usepackage[english]{babel}
\usepackage{geometry}
\usepackage{minted}
\usepackage{tabularx}
\usepackage{listings}
\usepackage{bussproofs}
\usepackage{multicol}
\usepackage{upgreek}
\usepackage{graphicx}
\usepackage{color}

%\lstset{language=JVMIS}
\usemintedstyle{borland}
%\geometry{left=10mm, top=20mm, bottom=10mm}
\geometry{left=10mm, top=10mm, bottom=10mm}
\definecolor{mygreen}{rgb}{0,0.53,0}

\expandafter\def\csname PYGborland@tok@na\endcsname{\def\PYGborland@tc##1{\textcolor[rgb]{1.00,0.00,0.00}{##1}}}

\pagenumbering{gobble}

\lstset{
  keywordstyle=\bfseries,
  morekeywords={
    aload,aload_0,areturn,arraylength,astore,astore_0,getfield,getstatic,goto,
    goto_w,iadd,iand,iconst_0,idiv,if_icmpeq,if_icmpge,if_icmpgt,if_icmple,
    if_icmplt,if_icmpne,ifeq,ifge,ifgt,ifle,iflt,ifne,ifnonnull,ifnull,iinc,
    iload, iload_0,imul,ineg,instanceof,invokedynamic,invokeinterface,
    invokespecial,invokestatic,invokevirtual,ior,ireturn,ishl,ishr,istore,
    istore_0,isub,iushr,ixor,ldc,new,nop,pop,pop2,putfield,putstatic,ret,return,
    swap
  },
  sensitive=false,
  morecomment=[l]{##},
  commentstyle=\bfseries\color{mygreen},
  %morecomment=[s]{/*}{*/},
  %morestring=[b]",
}

\begin{document}

\vspace{-1mm}

\section{A}
\begin{minted}[linenos,numberblanklines=false,tabsize=4,breaklines,breakindent=8]{haskell}
Beispielbegriff :: Kurzerklärung -- Lange erklärung 

Abiogenese -- Urzeugung, Erklärung der Entstehung von Leben aus Unbelebtem.

Actin -- Teil des Cytoskeletts, bildet spiralige Fäden, an denen Myosinmotoren entlangwandern, Baustein der Muskeln

Actomyosin -- Überbegriff für das actinbasierte Bewegungssystem bestehend aus einer Actin"straße" und einem Myosin"motor"

adaptive Radiation -- schnelle Artbildung durch Besiedlung verschiedener ökologischer Nischen in einem neuen Lebensraum

Aggregation -- Entstehung von Vielzelligkeit durch "Haufenbildung" einzelliger Formen

aktiver Transport -- Transport eines Moleküls über eine Membran entgegen dem chemischen Gradienten unter Aufwand von Energie

Aktualitätsprinzip -- Rekonstruktion der geologischen Vorzeit aufgrund der Gesetze der Jetztzeit (geht auf Lyell zurück)

Allel -- Variante eines Gens, z.B. bei einer Mutation entsteht aus dem Wildtyp- ein Mutanten-Allel

allopatrisch -- Variante der Artbildung, bei der die Subpopulation zuerst geographisch getrennt werden und danach genetisch allmählich so unterschiedlich werden, dass irgendwann eine Fortpflanzungsbarriere entsteht.

Allopolyploidie -- Hybridisierung mit anschliessender Verdopplung des Genoms, so dass eine fertile Nachkommenschaft entsteht. Häufiger Mechanismus der pflanzlichen Artbildung

Altruismus -- Gegenstück zu Egoismus, ein Individuum handelt zum Wohle der Anderen

amphipathisch -- Moleküle, die hydrophobe und hydrophile Bereiche vereinigen (z.B. Membranlipide)

Analogie -- eine durch ähnliche Funktion entstandene Ähnlichkeit der Gestalt (Synonym: Konvergenz), Gegensatz zur Homologie, wo die Ähnlichkeit auf Verwandtschaft (Abstammung) zurückgeht.

Aneuploidie -- einzelne Chromosomen sind in einer vom üblichen Chromosomensatz abweichenden Anzahl vorhanden, also entweder öfter, weniger oft oder ganz fehlen

Anticodon -- komplementäre Sequenz aus drei Nucleotiden auf der tRNS, das sich mit dem Codon auf der mRNS paart.

Aquaporine -- Carrier für Wasser, erhöhen die Wasserpermeabilität von Membranen für schnelle Transportreaktionen

Archaeen -- drittes Reich der Lebewesen neben Pro- und Eukaryoten. Die Archaeen kommen in Extremstandorten vor und entsprechen wohl den ältesten Lebensformen.

Artkonstanz -- die Vorstellung, dass Arten unveränderlich sind, häufig verbunden mit der Vorstellung, dass sie in dieser Form von einem göttlichen Wesen geschaffen worden seien. Es gibt christliche, aber auch muslimische Formen dieser kreationistischen Vorstellung.

Atmungskette -- Komplex aus Membranproteinen in der inneren Mitochondrienmembran, wo bei der Atmung ein Protonengradient entsteht, der dann die ATP-Synthase treibt.

ATP-Synthase -- eines der wichtigsten Proteine dieser Erde, das an der inneren Mitochondrienmembran oder der Thylakoidmembran sitzt und einen elektrischen Gradienten (aus Protonen) dazu benutzt, das energiereiche ATP herzustellen, mit dem zahlreiche energieverbrauchende Reaktionen des Lebens angetrieben werden.
    
\end{minted}
\newpage


\section{B}
\begin{minted}[linenos,numberblanklines=false,tabsize=4,breaklines,breakindent]{haskell}
Benthos -- festsitzende Flora der Gewässer, wegen der Abhängigkeit vom Licht auf die Küstenlinie beschränkt (Gegenbegriff: Plankton)

Binomiale Nomenklatur -- Bezeichnung aller Lebensformen über einen Gattungs- und einen Artnamen. Von Carl v. Linné allgemein eingeführt.

Biofilmtheorie -- Theorie, dass Leben an katalytische wirksamen Oberflächen von Pyriten entstand (alternativ zur Ursuppentheorie von Oparin und Haldane)

Biogenetisches Grundgesetz -- Modell von Häckel, wonach ein Lebewesen in der Individualentwicklung die Stammesentwicklung wiederholt

Barr Körperchen :: --
\end{minted}
\newpage


\section{C}
\begin{minted}[linenos,numberblanklines=false,tabsize=4,breaklines,breakindent]{haskell}
Carrier -- Membranproteine, die den Durchfluss von Molekülen ermöglichen, die sonst nicht hindurchgelangen würden

Chitin -- Baustoff der Zellwand von Pilzen und Insekten, Polymer, das aus Acetyl-Glucosamin besteht

Chlorophyll -- Blattgrün, Pflanzenfarbstoff, besteht aus einem hydrophilen Porphyrinring, wo die Lichtreaktion der Photosynthese stattfindet und einem hydrophoben Phytolschwanz, mit dem der Farbstoff an der Thylakoidmembran verankert ist.

Chromatin -- verpackte DNS im Zellkern, zum Ablesen (Transkription) muss sie ausgepackt werden, Name von chromos = Farbe, da man Chromatin histochemisch anfärben kann.

Codon -- "Wort" auf der mRNS aus drei Nucleotid"buchstaben", das eine Aminosäure "bedeutet"

Cristae -- Einstülpungen der inneren Mitochondrienmembran, wo die ATP-Synthase sitzt.

Cyclin -- Protein, das in einem Komplex mit einer Kinase den Zellzyklus weiterführt. Nach dem Zellzyklusschritt wird es abgebaut, um eine Dauerteilung zu verhindern.
\end{minted}
\newpage


\section{D}
\begin{minted}[linenos,numberblanklines=false,tabsize=4,breaklines,breakindent]{haskell}
degenerierter Code -- 20 Aminosäuren sind 64 mögliche Basentripletts zugeordnet, jedes Triplett entspricht eindeutig einer Aminosäure, aber eine Aminosäure kann von mehreren Tripletts kodiert sein.

Destruenten -- in Ökosystemen Organismen, die von den Überresten anderer Organismen leben.

Deszendenztheorie -- Erklärung von biologischer Ähnlichkeit aufgrund von Abstammung (widerspricht der Artkonstanz)

Deuterostomier -- Gruppe der Metazoa, wo der Urmund zum After wird, der Mund entsteht sekundär als Durchbruch gegenüber. Die "Wirbeltiere"

differentielle Genexpression -- Modell der Entwicklung, wonach alle Zellen zwar denselben Satz an Genen erhalten, aber unterschiedliche Teile davon aktivieren.
\end{minted}
\newpage


\section{E}
\begin{minted}[linenos,numberblanklines=false,tabsize=4,breaklines,breakindent]{haskell}
Egoistisches Gen -- Konzept von Richard Dawkins, das die Evolution von Altruismus erklären soll. Gene, die ihre eigene Vermehrung fördern, indem sie altruistisches Verhalten fördern, "handeln" im eigenen Interesse und werden von Dawkins als "egoistisch" (selfish) bezeichnet. Durch die sprachliche Psychologisierung der Genetik hat dieses Konzept viel Verwirrung gestiftet.

Ektoderm -- Außenschicht des Metazoen-Keims, dazu zählen Haut, Nervengewebe, Magen. Gegensatz Endoderm und Mesoderm

Endoplasmatisches Reticulum -- Membransystem, das das Innere eukaryotischer Zellen durchzieht. Stapel des ER bilden die Kernhülle, vom ER knospen Vesikel ab, die Stoffe zum Golgi-Apparat bringen, von wo sie weiter transportiert werden

Endosomen -- Vesikel, die durch Einstülpung an der Zellmembran entstehen und Dinge in die Zelle aufnehmen.

Endosymbiontentheorie -- auf Mereschkowskij und Margulis zurückgehende Theorie, dass Mitochondrien und Plastiden durch Endosymbiose von freilebenden Prokaryoten entstanden sind, dies erklärt die eigene (prokaryotische) DNS dieser Organellen, ihre eigenen (prokaryotischen) Ribosomen und ihre doppelte Membran.

Endproduktrepression -- Fall der Gensteuerung, bei dem ein Operon durch das (gebildete) Endprodukt abgeschaltet wird. Beispiel: Tryptophan-Synthese

Epigenese -- Modell der Entwicklung, wo die Gestalt des Organismus erst durch Wechselwirkungen der Teile des Embryo entsteht.

Epigenetik -- Die Fähigkeit eines Gens zur Expression ändert sich abhängig davon, wie oft es zuvor exprimiert wurde (eine Art "Gentraining"). Zellulär liegen Änderungen der Kernarchitektur, Methylierung der DNS und Deacetylierung der Histone zugrunde. Kann in manchen Fällen auch an die nächste Generation weitergegeben werden.

Eukaryoten -- Organismen, bei denen die DNS in einem Zellkern vom Cytoplasma getrennt ist: Pilze, Pflanzen und Tiere

Fluid -- Mosaic Model von Singer und Nicholson beschreibt die Membran als flüssige Matrix, worin Proteine mosaikartig eingebettet sind.
\end{minted}
\newpage


\section{F}
\begin{minted}[linenos,numberblanklines=false,tabsize=4,breaklines,breakindent]{haskell}

\end{minted}
\newpage


\section{G}
\begin{minted}[linenos,numberblanklines=false,tabsize=4,breaklines,breakindent]{haskell}
Gen -- Genetischer Faktor (Region der DNA), der Merkmal bestimmt

G0-Phase -- zeitweise oder stabile Unterbrechung des Zellzyklus (G steht für gap), wenn eine Zelle sich differenziert (Nervenzellen sind in G0)

Generation -- Folge von >1 Mitose zwischen Meiose und Befruchtung

Generationswechsel -- regelmäßige Abfolge von Generationen, die sich in ihrer Fortpflanzung oder Kernphase unterscheiden (für Pflanzen die Regel).

genetic drift -- zufällige Einflüsse auf die genetische Zusammensetzung einer Population, die nichts mit fitness zu tun haben. Spielt vor allem bei kleinen Populationen eine Rolle.

Glucose -- Baustein von Stärke und Cellulose, kann durch Wasserabspaltung verknüpft werden. Wenn die OH-Gruppe dabei nach unten zeigt, entsteht eine a-Verknüpfung (die Zuckerreste sind gleich orientiert, es entsteht eine Spirale: Amylose), wenn sie nach oben zeigt, entsteht eine b-Verknüpfung (jeder zweite Zuckerrest ist gedreht, es entsteht eine gestreckte Konfiguration, Cellulose).

Glycocalyx -- "Zuckerkelch", die zuckerhaltige Außenschicht tierischer Zellen (entsteht durch Glycolipide und Glycoproteine)

Golgi-Apparat -- dynamisches Organell aus Membranstapeln, wo Proteine, die vom ER kommen, prozessiert werden

Gradualismus -- die Auffassung, dass Evolution in kleinen Schritten gleichmäßig voranschreitet (Darwins Auffassung)

Gründereffekt -- Sonderform der genetic drift, bei der eine kleine Individuengruppe eine neue Population begründet, wodurch zufällige Abweichungen der Allelhäufigkeiten dann große Folgen haben

Gruppenselektion -- Selektion einer Gruppe von Individuen aufgrund einer Eigenschaft, die alle Individuen besitzen (z.B. Schwarmbildung)
\end{minted}
\newpage


\section{H}
\begin{minted}[linenos,numberblanklines=false,tabsize=4,breaklines,breakindent]{haskell}
Heterochronie -- zeitliche Verschiebung von Entwicklungsvorgängen, führt zu Veränderungen des Bauplans

Heterocyste -- stickstoff-fixierende Zelle der Cyanobacterien ("Blaualgen"), die keine Photosynthese
betreibt und von den Nachbarn mit Assimilaten versorgt werden muss.

Heterozygot -- Individuum mit zwei unterschiedlichen Allelen an einem bestimmten Lokus

Histone -- Proteine, die an die DNS binden und sie zu Knäueln (Nucleosomen) "verpacken"

Homologiekriterien -- Kriterien für eine evolutionäre Verwandtschaft - Lage, spezifische Qualität und Kontinuität

Homozygot -- Alle Allele eines Individuums Individuum mit zwei gleichen Allelen an einem bestimmten Lokus

Hybridisierung -- Paarung spiegelbildlicher DNS- oder RNS-Abschnitte zu stabilen Doppelsträngen
\end{minted}
\newpage


\section{I}
\begin{minted}[linenos,numberblanklines=false,tabsize=4,breaklines,breakindent]{haskell}
Intelligent Design -- pseudowissenschaftliche Spielart des Kreationismus, wo komplexe Strukturen als Spuren eines "intelligent design" betrachtet werden.

Intron -- nicht-kodierende Abschnitte der DNS, die in die kodierende Sequenz eingefügt sind und vor der Translation durch Spleissen herausgeschnitten werden. Diese Stücke steuern dann das Ablesen weiterer Gene, sind also eine Art Sprache für die gegenseitige Kommunikation der Gene.
\end{mi\newpage

nted}

\section{J}
\begin{minted}[linenos,numberblanklines=false,tabsize=4,breaklines,breakindent]{haskell}

\end{minted}
\newpage


\section{K}
\begin{minted}[linenos,numberblanklines=false,tabsize=4,breaklines,breakindent]{haskell}
K-Strategie -- ökologische Strategie von Organismen, die auf nachhaltige Nutzung der Resourcen bei langsamer Vermehrung abzielt

Katastrophentheorie -- Versuch von Cuvier, die Weiterentwicklung von Fossilien mit dem Dogma der Artkonstanz zu verbinden. Katastrophen seien von Neuschöpfung gefolgt

Kariotyp -- Alle Chromosomen eines Individuums

Kern-Plasma-Relation -- Schwellenwert von Plasma pro DNS, bei dessen Überschreiten eine Zellteilung eingeleitet wird.

Kinesin -- Klasse von Mikrotubulimotoren, die in der Regel durch Spaltung von ATP zum Pluspol des Mikrotubulus wandern

Koevolution -- gemeinsame evolutionäre Entwicklung von nicht verwandten Organismen. Paradebeispiel wären Blütenpflanzen und Hautflügler.

Kompartiment -- durch eine Membran umschlossener Reaktionsraum im Innern einer Zelle

Konsumenten -- in der Ökologie die Organismen, die durch Fressen anderer Organismen ihre Energie gewinnen

Kontinuitätsprinzip -- natura non saltat, Erklärung von Evolution über kleine Schritte

Konvergenz -- Ähnlichkeit von Strukturen, die nicht auf Verwandtschaft, sondern ähnlicher Funktion beruht

Kopplung -- wenn die Allele zweier Merkmale gemeinsam vererbt werden, spricht man von genetischer Kopplung. Deutung: die Gene für die beiden Merkmale liegen auf einem Chromosom.

Kormus -- Grundbauplan aller höheren Landpflanzen (Farne und Samenpflanzen) aus stabförmigen Elementen (Telomen), die aus Leitgewebe, Parenchym und Epidermis bestehen

Kulturelle Evolution -- Artentwicklung, die nicht auf genetischer Vererbung beruht, sondern auf Tradition. Für den Menschen wichtiger als biologische Evolution.
\end{minted}
\newpage


\section{L}
\begin{minted}[linenos,numberblanklines=false,tabsize=4,breaklines,breakindent]{haskell}
Lamarckismus -- die von Lamarck begründete Denkschule, wonach sich Lebewesen aufgrund eines Vervollkommnungstriebs anpassen und danach die erworbenen Eigenschaften weiter vererben

Latenzproblem -- Genetisches Problem, wie man erklärt, dass bei Kindern "verdeckte" Eigenschaften der Eltern auftreten. Wurde durch Mendel/Kölreuter über 2 Anlagen erklärt

Lipid Raft -- "Fettflösschen", Bereiche der Membran mit einem hohen Gehalt an gesättigten Fettsäuren (daher verdickt), wo zahlreiche Signalmoleküle (z.B. Rezeptoren) versammelt sind.

Lokus -- Spezifische Ort auf einen Chromosom wo man bestimmte Gene (Allele) findet

Lyssenkoismus -- sowjetische Spielart des Neolamarckismus, wo Genetik als "bourgeois" abgelehnt und durch "Weitergabe erworbener Eigenschaften" ersetzt wird, bis 1956 dominant
\end{minted}
\newpage


\section{M}
\begin{minted}[linenos,numberblanklines=false,tabsize=4,breaklines,breakindent]{haskell}
Makroevolution -- Entstehung neuer Baupläne oder qualitativ neuer Strukturen, von Darwin komplett ausgeblendet

Meiose -- Sonderform der Mitose, wobei in der ersten Prophase Stücke von mütterlichen und väterlichen Chromosomen ausgetauscht werden und in der zweiten Teilung durch Ausfallen der DNS-Synthese die DNS-Menge halbiert wird. Kernstück der Sexualität

Mikroevolution -- Darwins Evolutionsmodell einer allmählichen Veränderungen in kleinen Schritten durch Variation und Selektion. Damit lässt sich nicht die Entstehung neuer Baupläne erklären.

Mikrotubuli -- rohrförmige Polymere aus Tubulin, an denen Kinesin und Dynein-Motoren entlanglaufen, Grundlage der Geißelbewegung

Miller-Experiment -- Experimentelle Simulation der Situation auf der Urerde, dabei entstanden aus einfachen Molekülen komplexe Biomoleküle wie Aminosäuren, Nucleotide und Zucker

Mimikry -- Nachahmung von Organismen durch andere Organismen zum Zwecke der Tarnung

missing link -- zumeist hypothetische Zwischenformen eines evolutionären Wandels, die laut Darwin ein Mosaik aus "alt" und "neu" darstellen sollten. 

Mitochondrien -- Zellorganellen, in denen die Atmung stattfindet, entstanden aus freilebenden Bakterien und haben daher eigene DNS und Ribosomen

Mitose -- Teilung des Zellkerns, in der Regel mit einer Teilung der ganzen Zelle (Cytokinese) verbunden.

Modularisierung -- Bauplan aller Lebewesen aus Elementen, die wiederholt und abgewandelt werden (Telome der Pflanzen, Segmente der Tiere, Gliedmaßen)

Musterbildung -- bei der Entwicklung die Herausbildung einer geordneten Anordung von Elementen

Mutation -- zufällige Veränderung der DNS-Sequenz, führt zumeist zu einer veränderten Abfolge der Aminosäuren im davon kodierten Protein

Myosin -- Motorprotein, das unter ATP-Spaltung entlang von Actin wandern kann. Grundlage der Muskelbewegung
\end{minted}
\newpage


\section{N}
\begin{minted}[linenos,numberblanklines=false,tabsize=4,breaklines,breakindent]{haskell}
Nische -- der "Beruf" eines Organismus in einem Ökosystem, also die Überlebensstrategie, die verfolgt wird.
\end{minted}
\newpage


\section{O}
\begin{minted}[linenos,numberblanklines=false,tabsize=4,breaklines,breakindent]{haskell}
ökologische Potenz -- Bandbreite eines Organismus, die aufgrund seiner Physiologie und der Konkurrenz mit anderen Arten möglich ist.

Operon -- Einheit aus mehreren Genen, die gemeinsam für eine Funktion notwendig sind und durch denselben Promotor gesteuert werden.

Osmose -- Volumenänderungen von membranumschlossenen Räumen (Zellen), weil Wasser durch die Membran kann, Salze oder Zucker aber nicht.
\end{minted}
\newpage


\section{P}
\begin{minted}[linenos,numberblanklines=false,tabsize=4,breaklines,breakindent]{haskell}
Panspermie-Hypothese -- Die Hypothese, dass das Leben über Meteoriten im Weltall verbreitet wurde. Löst die Frage nach der Entstehung von Leben nicht. 

Parazoa -- Schein-Vielzeller, Kolonien aus Einzelzellen, die nicht wirklich ein Ganzes bilden (z.B. Schwämme)

PCR -- Polymerase Ketten (chain) Reaktion. Wichtiges Verfahren der Molekularbiologie, bei dem bestimmte DNS-Stücke mithilfe von hybridisierenden kurzen Erkennungssequenzen (Oligonucleotidprimern) vervielfacht werden. Man benutzt dazu hitzestabile Polymerasen, die das DNS-Stück zwischen den Erkennungssequenzen verdoppeln. Danach wird die gebildete DNS neu aufgeschmolzen und der Schritt wiederholt.

Phenotyp -- Erscheinungsform eines Merkmales

Phloëm -- sprich Phlo-em, Leitelement der Kormophyten, das Assimilate von den Blättern zu den Verbrauchsorganen bringt.

Phospholipide -- Bausteine der Membran, bestehen aus einem Glycerolrest, wo zwei OH-Gruppen mit Fettsäuren, die dritte OH-Gruppe über eine Phosphatgruppe mit einem weiteren Molekül verestert ist.

Photosysteme -- Komplexe aus Chlorophyll, Proteinen und Antennenpigmenten (Carotinoiden) in der Thylakoidmembran, wo die Lichtreaktionen der Photosynthese stattfindet. Viele Chlorophylle sind zu Lichtsammelkomplexen vereinigt, wobei nach Anregung ein Strom ans Reaktionszentrum des Photosystems fließt und von dort die weiteren Membranvorgänge gestartet werden.

physiologische Potenz -- Bandbreite eines Organismus, die aufgrund seiner Physiologie möglich ist

Plankton -- frei schwimmende Mikroorganismen (Algen, Protisten, kleine Tiere)

Plasmodesmen -- plasmatische Verbindung pflanzlicher Zelle, mikroskopische sichtbar als "Tüpfel".

Plasmolyse -- Sonderfall der Osmose, wobei Zellen in einem hypertonischen Medium (Zucker- oder Salzlösung) Wasser verlieren und schrumpfen

Plastiden -- semiautonome Organellen pflanzlicher Zellen mit eigener DNS und eigenen Ribosomen, die für die Photosynthese (Chloroplasten) verantwortlich sind, aber auch noch andere Stoffwechselwege (Carotinoidsynthese, Stärkebildung) beherbergen. Diese anderen Stoffwechselwege können gewebsabhängig hochgeregelt sein, wodurch Chromoplasten und Amyloplasten entstehen. Die Plastiden leiten sich von prokaryotischen Cyanobakterien ab.

Polarität -- eine "Richtung" von Faktoren, die Entwicklung steuern

Polygenismus -- Idee u.a. von Agassiz, dass die Menschenrassen nicht verwandt, sondern unabhängig geschaffen worden seien (auf der Basis der Artkonstanz)

Präadaptation -- makroevolutionäres Konzept, wonach ein Merkmal schon vorbereitet wird, obwohl die Notwendigkeit erst später kommen wird. Wird inzwischen durch Funktionswechsel erklärt.

Präformation -- Modell der Entwicklung, wo die Gestalt des Organismus schon vorgeformt im Ei vorliegt

primäre Atmosphäre -- reduzierende (sauerstoff-freie) Atmosphäre der Urerde

Produzenten -- in ökologischen Systemen die Organismen, die photosynthetisch Energie binden

Prokaryoten -- Organismen, bei denen die DNS frei vorliegt, haben also keinen Zellkern - Bakterien und Cyanobakterien

Protostomier -- Tiergruppe, bei der der Urmund der Gastrula zum Mund wird - die "Wirbellosen"

Polymerase
\end{minted}
\newpage


\section{Q}
\begin{minted}[linenos,numberblanklines=false,tabsize=4,breaklines,breakindent]{haskell}

\end{minted}
\newpage


\section{R}
\begin{minted}[linenos,numberblanklines=false,tabsize=4,breaklines,breakindent]{haskell}
r-Strategie -- ökologische Strategie von Organismen, die auf schnelle Vermehrung und Überdauerung abzielt

Rekombination -- die Neukombination von Genen bei der Meiose

Replikation -- Verdopplung der DNS

Restriktionsenzyme -- Werkzeug der Molekularbiologie, Enzyme, die DNS gezielt an bestimmten Erkennungsstellen schneiden

Rhinogradentier -- in der Karlsruher Biologie entdeckte Organismengruppe, die über adaptive Radiation ein pazifisches Archipel besiedelte, leider ausgestorben.

Ribosom -- Organell, an dem im Cytoplasma die Translation stattfindet, besteht aus Protein und rRNS

RNS-Welt -- Hypothese, dass vor der DNS-Protein Welt von heute Lebensformen mit RNS als Erb- und Enyzmsubstanz vorherrschten. Die RNS-Natur von rRNS, mRNS und tRNS gilt als wichtiger Beleg für dieses Modell.
\end{minted}
\newpage


\section{S}
\begin{minted}[linenos,numberblanklines=false,tabsize=4,breaklines,breakindent]{haskell}
sekundäre Atmosphäre -- die oxidative Atmosphäre, wie wir sie heute haben, die aus der reduzierenden primären Atmosphäre infolge der Photosynthese entstanden ist

Selektion -- "Zuchtwahl", Auslese der Individuen, die sich fortpflanzen. Bei der Züchtung sucht der Züchter aus, wer sich fortpflanzt. Bei der "natürlichen Zuchtwahl" entscheiden die Umweltbedingungen, welche Individuen übrigbleiben und sich fortpflanzen.

Semipermeabilität -- Halbdurchlässigkeit von Membranen, Wasser und kleine apolare Moleküle gelangen hindurch, große Moleküle und Ionen nicht

sexuelle Selektion -- Auswahl des Geschlechtspartners durch das andere Geschlecht aufgrund von besonderen Merkmalen, die etwas über die fitness aussagen und zu einem sexuellen Dimorphismus führen.

Soma -- bei der Entwicklung entstehende Zell-Linien, die nicht an der Fortpflanzung teilnehmen und daher sterblich sind.

Sozialdarwinismus -- Übertragung darwinistischer Evolution auf die menschliche Gesellschaft

Spaltungsregel -- auch 2. Mendelgesetz: bei der Selbstung von Heterozygoten (Aa) spaltet sich die Nachkommenschaft nach 1:2:1 in AA, Aa und aa auf.

Speziation -- Artbildung, zumeist allopatrisch (nach geographischer Auftrennung einer Population)

Substrataktivierung -- Fall der Gensteuerung, bei dem ein Operon durch den (umzusetzenden) Ausgangsstoff angeschaltet wird. Beispiel: Lactose-Abbau

sympatrisch -- Variante der Artbildung, bei der die Subpopulation ohne vorausgehende geographische Trennung eine Fortpflanzungsbarriere ausbildet.

synaptonemaler Komplex -- molekularer Reißverschluss, der mütterliche und väterliche Chromosomen passgenau nebeneinander legt - Voraussetzung für die Meiose

Synthetische Theorie -- Erweiterung von Darwins Theorie um populationsgenetische (Speziation), entwicklungsbiologische (Embryologie) und zellbiologische (Endosymbiose) Elemente.
\end{minted}
\newpage


\section{T}
\begin{minted}[linenos,numberblanklines=false,tabsize=4,breaklines,breakindent]{haskell}
Telom -- Baumodul aller höheren Landpflanzen (Kormophyten) bestehend aus Leitbündel, Parenchym und Epidermis

Thylakoidmembran -- innere Membran des Chloroplasten, zu Stapeln (Grana) aufgeschichtet. Hier findet die Lichtreaktion der Photosynthese statt.

Transition -- Mutation, bei der der Basentyp (Purin oder Pyrimidin) beibehalten wird

Transkription -- Ablesen der mRNS von der DNS (passiert im Zellkern)

Translation -- Übersetzen der mRNS in Protein (passiert im Cytoplasma an den Ribosomen)

Transversion -- Mutation, bei der eine Purin- in eine Pyrimidinbase umgewandelt wird (oder umgekehrt).

Turgor -- Wanddruck der Pflanzenzellen, entsteht durch Osmose, wodurch sich die Zelle ausdehnt und auf die Wand einen Druck ausübt, der das Wachstum antreibt.

\end{minted}
\newpage


\section{U}
\begin{minted}[linenos,numberblanklines=false,tabsize=4,breaklines,breakindent]{haskell}
Uniformitätsregel -- auch 1. Mendelgesetz: bei der Kreuzung zweier homozygoter Eltern (AA mit aa) entstehen einheitliche Nachkommen (Aa)

\end{minted}
\newpage


\section{V}
\begin{minted}[linenos,numberblanklines=false,tabsize=4,breaklines,breakindent]{haskell}
Variation - die genetische Unterschiedlichkeit von Individuen einer Population, Voraussetzung für die Evolution

Vektoren -- ringförmige DNS-Stücke von Bakterien, auf denen einzelne Gene für bestimmte Eigenschaften weitergegeben werden können. Diese wurden für molekularbiologische Zwecke umgebaut, um Gene von Interesse vervielfältigen und auf andere Organismen übertragen zu können.
\end{minted}
\newpage


\section{W}
\begin{minted}[linenos,numberblanklines=false,tabsize=4,breaklines,breakindent]{haskell}

\end{minted}
\newpage


\section{X}
\begin{minted}[linenos,numberblanklines=false,tabsize=4,breaklines,breakindent]{haskell}
Xylem -- Wasserleitungsgewebe der Kormophyten, verholzt, transportiert Wasser und Mineralien von der Wurzel nach oben

\end{minted}
\newpage


\section{Y}
\begin{minted}[linenos,numberblanklines=false,tabsize=4,breaklines,breakindent]{haskell}

\end{minted}
\newpage


\section{Z}
\begin{minted}[linenos,numberblanklines=false,tabsize=4,breaklines,breakindent]{haskell}
Zuchtwahl -- alter Begriff für Selektion, auf Darwins Modell des Taubenzüchters beruhend

\end{minted}
\newpage




\end{document}